% \iffalse meta-comment
%
% Copyright (C) 2019 by Zangwei Zheng <zhengzangw@gmail.com>
%
% This file may be distributed and/or modified under the conditions of
% the LaTeX Project Public License, either version 1.3c of this license
% or (at your option) any later version. The latest version of this 
% license is in:
%
%   http://www.latex-project.org/lppl.txt
%
% and version 1.3c or later is part of all distributions of LaTeX
% version 2005/12/01 or later.
%
% \fi
%
% \iffalse
%<*driver>
%
%<*batchfile>
\input docstrip
\keepsilent

\usedir{tex/latex/njurepo}

\preamble

This is a generated file.

Copyright (C) 2019 by Zangwei Zheng <zhengzangw@gmail.com>

This file may be distributed and/or modified under the
conditions of the LaTeX Project Public License, either version 1.3c
of this license or (at your option) any later version.
The latest version of this license is in:
  http://www.latex-project.org/lppl.txt
and version 1.3c or later is part of all distributions of LaTeX
version 2005/12/01 or later.

\endpreamble
\askforoverwritefalse
\generate{\file{\jobname.cls}{\from{\jobname.dtx}{cls}}
          \file{dtx-style.sty}{\from{\jobname.dtx}{dtx-style}}
  }

\obeyspaces
\Msg{*************************************************************}
\Msg{*                                                           *}
\Msg{* To finish the installation you have to move the following *}
\Msg{* files into a directory searched by TeX:                   *}
\Msg{*                                                           *}
\Msg{*     njurepo.cls                                           *}
\Msg{*     njurepo.cfg                                           *}
\Msg{*                                                           *}
\Msg{* To produce the documentation run the file njurepo.dtx     *}
\Msg{* through LaTeX.                                            *}
\Msg{*                                                           *}
\Msg{* Happy TeXing!                                             *}
\Msg{*                                                           *}
\Msg{*************************************************************}

\endbatchfile

%</batchfile>
%
\ProvidesFile{njurepo.dtx}[2019/03/20 1.1.3 Nanjing University Report Template]
\documentclass{ltxdoc}
\usepackage{dtx-style}
    \EnableCrossrefs
    \CodelineIndex
    \RecordChanges
\begin{document}
    \DocInput{njurepo.dtx}
    \PrintChanges
    \PrintIndex
\end{document}
%</driver>
% \fi
%
% \CheckSum{0}
%
% \CharacterTable
%  {Upper-case    \A\B\C\D\E\F\G\H\I\J\K\L\M\N\O\P\Q\R\S\T\U\V\W\X\Y\Z
%   Lower-case    \a\b\c\d\e\f\g\h\i\j\k\l\m\n\o\p\q\r\s\t\u\v\w\x\y\z
%   Digits        \0\1\2\3\4\5\6\7\8\9
%   Exclamation   \!     Double quote  \"     Hash (number) \#
%   Dollar        \$     Percent       \%     Ampersand     \&
%   Acute accent  \'     Left paren    \(     Right paren   \)
%   Asterisk      \*     Plus          \+     Comma         \,
%   Minus         \-     Point         \.     Solidus       \/
%   Colon         \:     Semicolon     \;     Less than     \<
%   Equals        \=     Greater than  \>     Question mark \?
%   Commercial at \@     Left bracket  \[     Backslash     \\
%   Right bracket \]     Circumflex    \^     Underscore    \_
%   Grave accent  \`     Left brace    \{     Vertical bar  \|
%   Right brace   \}     Tilde         \~}
%
% \DoNotIndex{\newenvironment,\@bsphack,\@empty,\@esphack,\sfcode}
% \DoNotIndex{\addtocounter,\label,\let,\linewidth,\newcounter}
% \DoNotIndex{\noindent,\normalfont,\par,\parskip,\phantomsection}
% \DoNotIndex{\providecommand,\ProvidesPackage,\refstepcounter}
% \DoNotIndex{\RequirePackage,\setcounter,\setlength,\string,\strut}
% \DoNotIndex{\textbackslash,\texttt,\ttfamily,\usepackage}
% \DoNotIndex{\begin,\end,\begingroup,\endgroup,\par,\\}
% \DoNotIndex{\if,\ifx,\ifdim,\ifnum,\ifcase,\else,\or,\fi}
% \DoNotIndex{\let,\def,\xdef,\edef,\newcommand,\renewcommand}
% \DoNotIndex{\expandafter,\csname,\endcsname,\relax,\protect}
% \DoNotIndex{\Huge,\huge,\LARGE,\Large,\large,\normalsize}
% \DoNotIndex{\small,\footnotesize,\scriptsize,\tiny}
% \DoNotIndex{\normalfont,\bfseries,\slshape,\sffamily,\interlinepenalty}
% \DoNotIndex{\textbf,\textit,\textsf,\textsc}
% \DoNotIndex{\hfil,\par,\hskip,\vskip,\vspace,\quad}
% \DoNotIndex{\centering,\raggedright,\ref}
% \DoNotIndex{\c@secnumdepth,\@startsection,\@setfontsize}
% \DoNotIndex{\ ,\@plus,\@minus,\p@,\z@,\@m,\@M,\@ne,\m@ne}
% \DoNotIndex{\@@par,\DeclareOperation,\RequirePackage,\LoadClass}
% \DoNotIndex{\AtBeginDocument,\AtEndDocument}
%
% \changes{v1.0.0}{2019/01/22}{Initial version}
% \changes{v1.0.1}{2019/01/29}{Add more ability}
% \changes{v1.1.0}{2019/01/29}{Stable version}
% \changes{v1.1.1}{2019/02/20}{Fix little bugs}
% \changes{v1.1.2}{2019/03/20}{Fix some typo and add more predined macro. Polish manual}
% \changes{v1.1.3}{2019/04/24}{Fix some bugs and add three more packages}
%
% \GetFileInfo{\jobname.dtx}
%
% \def\indexname{索引}
% \def\glossaryname{修改记录}
% \IndexPrologue{\section{\indexname}}
% \GlossaryPrologue{\section{\glossaryname}}

% \title{\bfseries\color{violet}\njurepo: 南京大学本科生泛用报告}
% \author{郑奘巍 \\[5pt]\texttt{zhengzangw@gmail.com}}
% \date{\fileversion\ (\filedate)}
% \maketitle\thispagestyle{empty}
%
% \begin{abstract}\noindent
% 此宏包旨在建立一个免于配置的、指令相对简单的南京大学作业、实验报告通用模板。
% \end{abstract}
%
%
% \vskip2cm
% \def\abstractname{免责声明}
% \begin{abstract}
% \noindent
% \begin{enumerate}
% \item 本模板的发布遵守 \LaTeX\ Project Public License,使用前请认真阅读协议内
%   容。
% \item \textbf{本模板为作者自己通常使用的报告模板,与南京大学官方没有任何关系}。任何使用该宏包进行实验报告制作时,请\textbf{务必根据课程要求进行写作}。由于使用本模板而引起的作业验收问题,均与本模板作者无关。
% \item 本模板借鉴\njuthesis{}宏包的许多内容,需要稳定模板的同学也可以选择使用南京大学的\njuthesis{}宏包并自己进行配置。
% \item 任何个人或组织以本模板为基础进行修改、扩展而生成的新的专用模板,请严格遵
%   守 \LaTeX\ Project Public License 协议。由于违犯协议而引起的任何纠纷争端均与
%   本模板作者无关。
% \end{enumerate}
% \end{abstract}
%
% \clearpage
% \pagestyle{fancy}
% \begin{multicols}{2}[
%   \setlength{\columnseprule}{.4pt}
%   \setlength{\columnsep}{18pt}]
%   \tableofcontents
% \end{multicols}
% \clearpage
%
% \section{模板介绍}
% \njurepo\ (\textbf{N}an\textbf{jing} \textbf{U}niversity \LaTeX\ Versatile \textbf{Repo}rt Template)是为南京大学本科生设计的一个免于配置的作业、实验报告模板。希望它可以使你的作业/实验报告不会因形式上的缺陷导致评分的下降。
%
% 该文档格式基于 ctexbook, 主要完成了除了主体内容以外的几乎\textbf{全部}工作。同时,通过使用 Github 版本宏包,你还可以更好的管理自己的 $\LaTeX$ 文档。
%
% 本文档将尽量完整的介绍模板的使用方法,如有不清楚之处可以参考示例文档或者根据第 3.1 节说明提问,有兴趣者都可以参与完善此手册,也非常欢迎对代码的贡献。
% 
% \section{安装}
% \label{sec:installation}
% \subsection{CTAN}
% 本宏包已被收纳于 CTAN 中,凡安装完整版 Texlive 用户可直接使用:
% \begin{latex} 
% \usepackage{njurepo} 
% \end{latex}
%
% \subsection{Github}
% 想获得最新版本的 \njurepo 请前往 Github 主页下载:https://github.com/zhengzangw/njurepo 
% 
% 同时,面向 Github 版本的说明主要在 README.md 中,如果使用的是 Github 版本,请阅读 README.md
%
% \subsection{模板的组成}
% 下表列出了\njurepo 的主要文件及其功能介绍:
%
% \begin{longtable}{l|p{8cm}}
% \toprule
% {\heiti 文件(夹)} & {\heiti 功能描述}\\\midrule
% \endfirsthead
% \midrule
% {\heiti 文件(夹)} & {\heiti 功能描述}\\\midrule
% \endhead
% \endfoot
% \endlastfoot
% njurepo.ins & \textsc{DocStrip} 驱动文件(开发用) \\
% njurepo.dtx & \textsc{DocStrip} 源文件(开发用)\\\midrule
% resources/ & 资源路径\\
% resources/logos/ & 示例文档资源路径\\
% parts/ & 具体内容\\
% parts/examples/ & 示例文档具体内容\\
% ref/ & 参考文献和参考文献样式文件\\
% njurepo.cls & 模板类文件\\
% Makefile & 自动运行脚本 \\
% util.py & 实用管理工具 \\
% README.md & 说明文件 \\
% \textbf{njurepo.pdf} & 用户手册(本文档)\\ \bottomrule
% \end{longtable}
%
% \subsection{生成模板}
% 使用Makefile或\XeLaTeX 生成模板文件
% \begin{shell}
% make cls
% \end{shell}
%
% \subsection{生成论文}
% \subsubsection{latexmk}
% latexmk 命令支持全自动生成\LaTeX{}编写的文档,并且支持使用不同的工具链来进行生成,它会自动运行多次工具直到交叉引用都被解决。下面给出了一个用 latexmk 调用 xelatex 生成最终文档的示例:
% \begin{shell}
% latexmk -xelatex main
% \end{shell}
%
% \subsection{升级}
% 在github上下载最新版,运行:
% \begin{shell}
% make cls
% \end{shell}
%
% 生成新的类文件和配置文件即可。也可以直接拷贝 njurepo.cls,免去上面命令的执行。
% 
%
% \section{使用说明}
% \subsection{示例文件}
% 推荐从模板自带的示例文档入手,其中包括了论文写作用到的所有命令及其使用方法,只需要用自己的内容进行相应替换就可以。对于不清楚的命令可以查阅本手册。下面的例子描述了模板中章节的组织形式,来自于示例文档,具体内容可以参考模板附带的 main.tex 和 parts/examples/。
% \begin{latex}
% \documentclass[language=english,open=any]{njurepo}
% \begin{document}
% \frontmatter
% \njusetup{
  %******************************
  % 注意:
  %   1. 配置里面不要出现空行
  %   2. 不需要的配置信息可以删除
  %******************************
  %
  %=========
  % 中文信息
  %=========
  ctitle={清华大学学位论文 \LaTeX\ 模板\\使用示例文档},
  csubtitle={通用报告模板},
  cdegree={工学硕士},
  cdepartment={计算机科学与技术系},
  cmajor={计算机科学与技术},
  cauthor={薛瑞尼},
  stdid={12345678},
  csupervisor={郑纬民教授},
  cassosupervisor={陈文光教授}, % 副指导老师
  ccosupervisor={某某某教授}, % 联合指导老师
  % 日期自动使用当前时间,若需指定按如下方式修改:
  % cdate={超新星纪元},
  %
  %=========
  % 英文信息
  %=========
  etitle={An Introduction to \LaTeX{} Thesis Template of Nanjing University v\version},
  esubtitle={subtile here},
  %esubsubtitle={Sub},
  edegree={Doctor of Engineering},
  emajor={Computer Science and Technology},
  eauthor={Xue Ruini},
  esupervisor={Professor Zheng Weimin},
  eassosupervisor={Chen Wenguang},
  % 日期自动生成,若需指定按如下方式修改:
  % edate={December, 2005}
  %
  % 关键词用“英文逗号”分割
  ckeywords={\TeX, \LaTeX, CJK, 模板, 论文},
  ekeywords={\TeX, \LaTeX, CJK, template, thesis}
}


% % 定义中英文摘要和关键字
\begin{cabstract}
    论文的摘要是对论文研究内容和成果的高度概括。摘要应对论文所研究的问题及其研究目
    的进行描述,对研究方法和过程进行简单介绍,对研究成果和所得结论进行概括。摘要应
    具有独立性和自明性,其内容应包含与论文全文同等量的主要信息。使读者即使不阅读全
    文,通过摘要就能了解论文的总体内容和主要成果。
  
    论文摘要的书写应力求精确、简明。切忌写成对论文书写内容进行提要的形式,尤其要避
    免“第 1 章……;第 2 章……;……”这种或类似的陈述方式。
  
    本文介绍清华大学论文模板 \njuthesis{} 的使用方法。本模板符合学校的本科、硕士、
    博士论文格式要求。
  
    本文的创新点主要有:
    \begin{itemize}
      \item 用例子来解释模板的使用方法;
      \item 用废话来填充无关紧要的部分;
      \item 一边学习摸索一边编写新代码。
    \end{itemize}
  
    关键词是为了文献标引工作、用以表示全文主要内容信息的单词或术语。关键词不超过 5
    个,每个关键词中间用分号分隔。(模板作者注:关键词分隔符不用考虑,模板会自动处
    理。英文关键词同理。)
  \end{cabstract}
  
  \ckeywords{\TeX, \LaTeX, CJK, 模板, 论文}
  
  \begin{eabstract}
     An abstract of a dissertation is a summary and extraction of research work
     and contributions. Included in an abstract should be description of research
     topic and research objective, brief introduction to methodology and research
     process, and summarization of conclusion and contributions of the
     research. An abstract should be characterized by independence and clarity and
     carry identical information with the dissertation. It should be such that the
     general idea and major contributions of the dissertation are conveyed without
     reading the dissertation.
  
     An abstract should be concise and to the point. It is a misunderstanding to
     make an abstract an outline of the dissertation and words ``the first
     chapter'', ``the second chapter'' and the like should be avoided in the
     abstract.
  
     Key words are terms used in a dissertation for indexing, reflecting core
     information of the dissertation. An abstract may contain a maximum of 5 key
     words, with semi-colons used in between to separate one another.
  \end{eabstract}
  
  \ekeywords{\TeX, \LaTeX, CJK, template, thesis}
  
% \maketitlepage
% \makecover
% \makeabstract
% \tableofcontents
% \begin{denotation}[3cm]
\item[HPC] 高性能计算 (High Performance Computing)
\item[cluster] 集群
\item[Itanium] 安腾
\item[SMP] 对称多处理
\item[API] 应用程序编程接口
\item[PI] 聚酰亚胺
\item[MPI] 聚酰亚胺模型化合物,N-苯基邻苯酰亚胺
\item[PBI] 聚苯并咪唑
\item[MPBI] 聚苯并咪唑模型化合物,N-苯基苯并咪唑
\item[PY] 聚吡咙
\item[PMDA-BDA]	均苯四酸二酐与联苯四胺合成的聚吡咙薄膜
\item[$\Delta G$] 活化自由能 (Activation Free Energy)
\item[$\chi$] 传输系数 (Transmission Coefficient)
\item[$E$] 能量
\item[$m$] 质量
\item[$c$] 光速
\item[$P$] 概率
\item[$T$] 时间
\item[$v$] 速度
\item[劝学] 君子曰:学不可以已。青,取之于蓝,而青于蓝;冰,水为之,而寒于水。木
  直中绳。輮以为轮,其曲中规。虽有槁暴,不复挺者,輮使之然也。故木受绳则直,金就
  砺则利,君子博学而日参省乎己,则知明而行无过矣。吾尝终日而思矣,不如须臾之所学
  也;吾尝跂而望矣,不如登高之博见也。登高而招,臂非加长也,而见者远;顺风而呼,
  声非加疾也,而闻者彰。假舆马者,非利足也,而致千里;假舟楫者,非能水也,而绝江
  河,君子生非异也,善假于物也。积土成山,风雨兴焉;积水成渊,蛟龙生焉;积善成德,
  而神明自得,圣心备焉。故不积跬步,无以至千里;不积小流,无以成江海。骐骥一跃,
  不能十步;驽马十驾,功在不舍。锲而舍之,朽木不折;锲而不舍,金石可镂。蚓无爪牙
  之利,筋骨之强,上食埃土,下饮黄泉,用心一也。蟹六跪而二螯,非蛇鳝之穴无可寄托
  者,用心躁也。—— 荀况
\end{denotation}



% % 也可以使用 nomencl 宏包:

% \printnomenclature[3cm]

% \nomenclature{HPC}{高性能计算 (High Performance Computing)}
% \nomenclature{cluster}{集群}
% \nomenclature{Itanium}{安腾}
% \nomenclature{SMP}{对称多处理}
% \nomenclature{API}{应用程序编程接口}
% \nomenclature{PI}{聚酰亚胺}
% \nomenclature{MPI}{聚酰亚胺模型化合物,N-苯基邻苯酰亚胺}
% \nomenclature{PBI}{聚苯并咪唑}
% \nomenclature{MPBI}{聚苯并咪唑模型化合物,N-苯基苯并咪唑}
% \nomenclature{PY}{聚吡咙}
% \nomenclature{PMDA-BDA}{均苯四酸二酐与联苯四胺合成的聚吡咙薄膜}
% \nomenclature{$\Delta G$}{活化自由能 (Activation Free Energy)}
% \nomenclature{$\chi$}{传输系数 (Transmission Coefficient)}
% \nomenclature{$E$}{能量}
% \nomenclature{$m$}{质量}
% \nomenclature{$c$}{光速}
% \nomenclature{$P$}{概率}
% \nomenclature{$T$}{时间}
% \nomenclature{$v$}{速度}

% \mainmatter
% \maketitle
% \input{parts/examples/problemsolving}
% \input{parts/examples/mathpro}
% \chapter{带 English 的标题}
\label{cha:intro}

这是 NJUrepo 的示例文档,基本上覆盖了模板中所有格式的设置。建议大家在使用模板之前,除了阅读《NJUrepo用户手册》,这个示例文档也最好能看一看。

小老鼠偷吃热凉粉;短长虫环绕矮高粱\footnote{韩愈(768-824),字退之,河南河阳(
  今河南孟县)人,自称郡望昌黎,世称韩昌黎。幼孤贫刻苦好学,德宗贞元八年进士。曾
  任监察御史,因上疏请免关中赋役,贬为阳山县令。后随宰相裴度平定淮西迁刑部侍郎,
  又因上表谏迎佛骨,贬潮州刺史。做过吏部侍郎,死谥文公,故世称韩吏部、韩文公。是
  唐代古文运动领袖,与柳宗元合称韩柳。诗力求险怪新奇,雄浑重气势。}。


\section{封面相关}
封面的例子请参看 \texttt{cover.tex}。主要符号表参看 \texttt{denotation.tex},附录和
个人简历分别参看 \texttt{appendix01.tex} 和 \texttt{resume.tex}。里面的命令都很直
观,一看即会\footnote{你说还是看不懂?怎么会呢?}。

\section{字体命令}
\label{sec:first}

苏轼(1037-1101),北宋文学家、书画家。字子瞻,号东坡居士,眉州眉山(今属四川)人
。苏洵子。嘉佑进士。神宗时曾任祠部员外郎,因反对王安石新法而求外职,任杭州通判,
知密州、徐州、湖州。后以作诗“谤讪朝廷”罪贬黄州。哲宗时任翰林学士,曾出知杭州、
颖州等,官至礼部尚书。后又贬谪惠州、儋州。北还后第二年病死常州。南宋时追谥文忠。
与父洵弟辙,合称“三苏”。在政治上属于旧党,但也有改革弊政的要求。其文汪洋恣肆,
明白畅达,为“唐宋八大家”之一。  其诗清新豪健,善用夸张比喻,在艺术表现方面独具
风格。少数诗篇也能反映民间疾苦,指责统治者的奢侈骄纵。词开豪放一派,对后代很有影
响。《念奴娇·赤壁怀古》、《水调歌头·丙辰中秋》传诵甚广。

{\kaishu 坡仙擅长行书、楷书,取法李邕、徐浩、颜真卿、杨凝式,而能自创新意。用笔丰腴
  跌宕,有天真烂漫之趣。与蔡襄、黄庭坚、米芾并称“宋四家”。能画竹,学文同,也喜
  作枯木怪石。论画主张“神似”,认为“论画以形似,见与儿童邻”;高度评价“诗中有
  画,画中有诗”的艺术造诣。诗文有《东坡七集》等。存世书迹有《答谢民师论文帖》、
  《祭黄几道文》、《前赤壁赋》、《黄州寒食诗帖》等。  画迹有《枯木怪石图》、《
  竹石图》等。}

{\fangsong 易与天地准,故能弥纶天地之道。仰以观於天文,俯以察於地理,是故知幽明之故。原
  始反终,故知死生之说。精气为物,游魂为变,是故知鬼神之情状。与天地相似,故不违。
  知周乎万物,而道济天下,故不过。旁行而不流,乐天知命,故不忧。安土敦乎仁,故
  能爱。范围天地之化而不过,曲成万物而不遗,通乎昼夜之道而知,故神无方而易无体。}

% 非本科生一般用不到幼圆与隶书字体。需要的同学请查看 ctex 文档。
{\ifcsname youyuan\endcsname\youyuan\else[无 \cs{youyuan} 字体。]\fi 有天地,然后
  万物生焉。盈天地之间者,唯万物,故受之以屯;屯者盈也,屯者物之始生也。物生必蒙,
  故受之以蒙;蒙者蒙也,物之穉也。物穉不可不养也,故受之以需;需者饮食之道也。饮
  食必有讼,故受之以讼。讼必有众起,故受之以师;师者众也。众必有所比,故受之以比;
  比者比也。比必有所畜也,故受之以小畜。物畜然后有礼,故受之以履。}

{\heiti 履而泰,然后安,故受之以泰;泰者通也。物不可以终通,故受之以否。物不可以终
  否,故受之以同人。与人同者,物必归焉,故受之以大有。有大者不可以盈,故受之以谦。
  有大而能谦,必豫,故受之以豫。豫必有随,故受之以随。以喜随人者,必有事,故受
  之以蛊;蛊者事也。}

{\ifcsname lishu\endcsname\lishu\else[无 \cs{lishu} 字体。]\fi 有事而后可大,故受
  之以临;临者大也。物大然后可观,故受之以观。可观而后有所合,故受之以噬嗑;嗑者
  合也。物不可以苟合而已,故受之以贲;贲者饰也。致饰然后亨,则尽矣,故受之以剥;
  剥者剥也。物不可以终尽,剥穷上反下,故受之以复。复则不妄矣,故受之以无妄。}

{\songti 有无妄然后可畜,故受之以大畜。物畜然后可养,故受之以颐;颐者养也。不养则不
  可动,故受之以大过。物不可以终过,故受之以坎;坎者陷也。陷必有所丽,故受之以
  离;离者丽也。}

\section{表格样本}
\label{chap1:sample:table}

\subsection{基本表格}
\label{sec:basictable}

模板中关于表格的宏包有三个:\pkg{booktabs}、\pkg{array} 和 \pkg{longtabular},命
令有一个 \cs{hlinewd}。三线表可以用 \pkg{booktabs} 提供
的 \cs{toprule}、\cs{midrule} 和 \cs{bottomrule}。它们与 \pkg{longtable} 能很好的
配合使用。如果表格比较简单的话可以直接用命令 \cs{hlinewd}\marg{width} 控制。
\begin{table}[htb]
  \centering
  \begin{minipage}[t]{0.8\linewidth} % 如果想在表格中使用脚注,minipage是个不错的办法
  \caption[模板文件]{模板文件。如果表格的标题很长,那么在表格索引中就会很不美
    观,所以要像 chapter 那样在前面用中括号写一个简短的标题。这个标题会出现在索
    引中。}
  \label{tab:template-files}
    \begin{tabularx}{\linewidth}{lX}
      \toprule[1.5pt]
      {\heiti 文件名} & {\heiti 描述} \\\midrule[1pt]
      thuthesis.ins & \LaTeX{} 安装文件,\textsc{DocStrip}\footnote{表格中的脚注} \\
      thuthesis.dtx & 所有的一切都在这里面\footnote{再来一个}。\\
      thuthesis.cls & 模板类文件。\\
      thuthesis.cfg & 模板配置文。cls 和 cfg 由前两个文件生成。\\
      thuthesis-numeric.bst    & 参考文献 BIB\TeX\ 样式文件。\\
      thuthesis-author-year.bst    & 参考文献 BIB\TeX\ 样式文件。\\
      thuthesis.sty   & 常用的包和命令写在这里,减轻主文件的负担。\\
      \bottomrule[1.5pt]
    \end{tabularx}
  \end{minipage}
\end{table}

首先来看一个最简单的表格。表 \ref{tab:template-files} 列举了本模板主要文件及其功
能。请大家注意三线表中各条线对应的命令。这个例子还展示了如何在表格中正确使用脚注。
由于 \LaTeX{} 本身不支持在表格中使用 \cs{footnote},所以我们不得不将表格放在
小页中,而且最好将表格的宽度设置为小页的宽度,这样脚注看起来才更美观。

\subsection{复杂表格}
\label{sec:complicatedtable}

我们经常会在表格下方标注数据来源,或者对表格里面的条目进行解释。前面的脚注是一种
不错的方法,如果不喜欢脚注,可以在表格后面写注释,比如表~\ref{tab:tabexamp1}。
\begin{table}[htbp]
  \centering
  \caption{复杂表格示例 1。这个引用 \cite{tex} 不会导致编号混乱。}
  \label{tab:tabexamp1}
  \begin{minipage}[t]{0.8\textwidth}
    \begin{tabularx}{\linewidth}{|l|X|X|X|X|}
      \hline
      \multirow{2}*{\diagbox[width=5em]{x}{y}} & \multicolumn{2}{c|}{First Half} & \multicolumn{2}{c|}{Second Half}\\\cline{2-5}
      & 1st Qtr &2nd Qtr&3rd Qtr&4th Qtr \\ \hline
      East$^{*}$ &   20.4&   27.4&   90&     20.4 \\
      West$^{**}$ &   30.6 &   38.6 &   34.6 &  31.6 \\ \hline
    \end{tabularx}\\[2pt]
    \footnotesize 注:数据来源《NJUrepo{} 使用手册》。\\
    *:东部\\
    **:西部
  \end{minipage}
\end{table}

此外,表~\ref{tab:tabexamp1} 同时还演示了另外两个功能:1)通过 \pkg{tabularx} 的
 \texttt{|X|} 扩展实现表格自动放大;2)通过命令 \cs{diagbox} 在表头部分
插入反斜线。

为了使我们的例子更接近实际情况,我会在必要的时候插入一些“无关”文字,以免太多图
表同时出现,导致排版效果不太理想。第一个出场的当然是我的最爱:风流潇洒、骏马绝尘、
健笔凌云的{\heiti 李太白}了。

李白,字太白,陇西成纪人。凉武昭王暠九世孙。或曰山东人,或曰蜀人。白少有逸才,志
气宏放,飘然有超世之心。初隐岷山,益州长史苏颋见而异之,曰:“是子天才英特,可比
相如。”天宝初,至长安,往见贺知章。知章见其文,叹曰:“子谪仙人也。”言于明皇,
召见金銮殿,奏颂一篇。帝赐食,亲为调羹,有诏供奉翰林。白犹与酒徒饮于市,帝坐沉香
亭子,意有所感,欲得白为乐章,召入,而白已醉。左右以水颒面,稍解,援笔成文,婉丽
精切。帝爱其才,数宴见。白常侍帝,醉,使高力士脱靴。力士素贵,耻之,摘其诗以激杨
贵妃。帝欲官白,妃辄沮止。白自知不为亲近所容,恳求还山。帝赐金放还。乃浪迹江湖,
终日沉饮。永王璘都督江陵,辟为僚佐。璘谋乱,兵败,白坐长流夜郎,会赦得还。族人阳
冰为当涂令,白往依之。代宗立,以左拾遗召,而白已卒。文宗时,诏以白歌诗、裴旻剑舞、
张旭草书为三绝云。集三十卷。今编诗二十五卷。\hfill —— 《全唐诗》诗人小传

浮动体的并排放置一般有两种情况:1)二者没有关系,为两个独立的浮动体;2)二者隶属
于同一个浮动体。对表格来说并排表格既可以像图~\ref{tab:parallel1}、
图~\ref{tab:parallel2} 使用小页环境,也可以如图~\ref{tab:subtable} 使用子表格来做。
图的例子参见第~\ref{sec:multifig} 节。

\begin{table}[htbp]
\noindent\begin{minipage}{0.5\textwidth}
\centering
\caption{第一个并排子表格}
\label{tab:parallel1}
\begin{tabular}{p{2cm}p{2cm}}
\toprule[1.5pt]
111 & 222 \\\midrule[1pt]
222 & 333 \\\bottomrule[1.5pt]
\end{tabular}
\end{minipage}%
\begin{minipage}{0.5\textwidth}
\centering
\caption{第二个并排子表格}
\label{tab:parallel2}
\begin{tabular}{p{2cm}p{2cm}}
\toprule[1.5pt]
111 & 222 \\\midrule[1pt]
222 & 333 \\\bottomrule[1.5pt]
\end{tabular}
\end{minipage}
\end{table}

然后就是忧国忧民,诗家楷模杜工部了。杜甫,字子美,其先襄阳人,曾祖依艺为巩令,因
居巩。甫天宝初应进士,不第。后献《三大礼赋》,明皇奇之,召试文章,授京兆府兵曹参
军。安禄山陷京师,肃宗即位灵武,甫自贼中遁赴行在,拜左拾遗。以论救房琯,出为华州
司功参军。关辅饥乱,寓居同州同谷县,身自负薪采梠,餔糒不给。久之,召补京兆府功曹,
道阻不赴。严武镇成都,奏为参谋、检校工部员外郎,赐绯。武与甫世旧,待遇甚厚。乃于
成都浣花里种竹植树,枕江结庐,纵酒啸歌其中。武卒,甫无所依,乃之东蜀就高適。既至
而適卒。是岁,蜀帅相攻杀,蜀大扰。甫携家避乱荆楚,扁舟下峡,未维舟而江陵亦乱。乃
溯沿湘流,游衡山,寓居耒阳。卒年五十九。元和中,归葬偃师首阳山,元稹志其墓。天宝
间,甫与李白齐名,时称李杜。然元稹之言曰:“李白壮浪纵恣,摆去拘束,诚亦差肩子美
矣。至若铺陈终始,排比声韵,大或千言,次犹数百,词气豪迈,而风调清深,属对律切,
而脱弃凡近,则李尚不能历其藩翰,况堂奥乎。”白居易亦云:“杜诗贯穿古今,  尽工尽
善,殆过于李。”元、白之论如此。盖其出处劳佚,喜乐悲愤,好贤恶恶,一见之于诗。而
又以忠君忧国、伤时念乱为本旨。读其诗可以知其世,故当时谓之“诗史”。旧集诗文共六
十卷,今编诗十九卷。

\begin{table}[htbp]
\centering
\caption{并排子表格}
\label{tab:subtable}
\subcaptionbox{第一个子表格}
{
\begin{tabular}{p{2cm}p{2cm}}
\toprule[1.5pt]
111 & 222 \\\midrule[1pt]
222 & 333 \\\bottomrule[1.5pt]
\end{tabular}
}
\hskip2cm
\subcaptionbox{第二个子表格}
{
\begin{tabular}{p{2cm}p{2cm}}
\toprule[1.5pt]
111 & 222 \\\midrule[1pt]
222 & 333 \\\bottomrule[1.5pt]
\end{tabular}
}
\end{table}

不可否认 \LaTeX{} 的表格功能没有想象中的那么强大,不过只要足够认真,足够细致,
同样可以排出来非常复杂非常漂亮的表格。请参看表~\ref{tab:tabexamp2}。
\begin{table}[htbp]
  \centering\dawu[1.3]
  \caption{复杂表格示例 2}
  \label{tab:tabexamp2}
  \begin{tabular}[c]{|m{1.5cm}|c|c|c|c|c|c|}\hline
    \multicolumn{2}{|c|}{Network Topology} & \# of nodes &
    \multicolumn{3}{c|}{\# of clients} & Server \\\hline
    GT-ITM & Waxman Transit-Stub & 600 &
    \multirow{2}{1em}{2\%}&
    \multirow{2}{1.5em}{10\%}&
    \multirow{2}{1.5em}{50\%}&
    \multirow{2}{1.2in}{Max. Connectivity}\\\cline{1-3}
    \multicolumn{2}{|c|}{Inet-2.1} & 6000 & & & &\\\hline
    \multirow{2}{1.5cm}{Xue} & Rui  & Ni &\multicolumn{4}{c|}{\multirow{2}*{NJUrepo}}\\\cline{2-3}
    & \multicolumn{2}{c|}{ABCDEF} &\multicolumn{4}{c|}{} \\\hline
\end{tabular}
\end{table}

最后就是清新飘逸、文约意赅、空谷绝响的王大侠了。王维,字摩诘,河东人。工书画,与
弟缙俱有俊才。开元九年,进士擢第,调太乐丞。坐累为济州司仓参军,历右拾遗、监察御
史、左补阙、库部郎中,拜吏部郎中。天宝末,为给事中。安禄山陷两都,维为贼所得,服
药阳喑,拘于菩提寺。禄山宴凝碧池,维潜赋诗悲悼,闻于行在。贼平,陷贼官三等定罪,
特原之,责授太子中允,迁中庶子、中书舍人。复拜给事中,转尚书右丞。维以诗名盛于开
元、天宝间,宁薛诸王驸马豪贵之门,无不拂席迎之。得宋之问辋川别墅,山水绝胜,与道
友裴迪,浮舟往来,弹琴赋诗,啸咏终日。笃于奉佛,晚年长斋禅诵。一日,忽索笔作书
数纸,别弟缙及平生亲故,舍笔而卒。赠秘书监。宝应中,代宗问缙:“朕常于诸王坐闻维
乐章,今存几何?”缙集诗六卷,文四卷,表上之。敕答云,卿伯氏位列先朝,名高希代。
抗行周雅,长揖楚辞。诗家者流,时论归美。克成编录,叹息良深。殷璠谓维诗词秀调雅,
意新理惬。在泉成珠,著壁成绘。苏轼亦云:“维诗中有画,画中有诗也。”今编诗四卷。

要想用好论文模板还是得提前学习一些 \TeX/\LaTeX{}的相关知识,具备一些基本能力,掌
握一些常见技巧,否则一旦遇到问题还真是比较麻烦。我们见过很多这样的同学,一直以来
都是使用 Word 等字处理工具,以为 \LaTeX{}模板的用法也应该类似,所以就沿袭同样的思
路来对待这种所见非所得的排版工具,结果被折腾的焦头烂额,疲惫不堪。

如果您要排版的表格长度超过一页,那么推荐使用 \pkg{longtable} 或者 \pkg{supertabular}
宏包,模板对 \pkg{longtable} 进行了相应的设置,所以用起来可能简单一些。
表~\ref{tab:performance} 就是 \pkg{longtable} 的简单示例。
\begin{longtable}[c]{c*{6}{r}}
\caption{实验数据}\label{tab:performance}\\
\toprule[1.5pt]
 测试程序 & \multicolumn{1}{c}{正常运行} & \multicolumn{1}{c}{同步} & \multicolumn{1}{c}{检查点} & \multicolumn{1}{c}{卷回恢复}
& \multicolumn{1}{c}{进程迁移} & \multicolumn{1}{c}{检查点} \\
& \multicolumn{1}{c}{时间 (s)}& \multicolumn{1}{c}{时间 (s)}&
\multicolumn{1}{c}{时间 (s)}& \multicolumn{1}{c}{时间 (s)}& \multicolumn{1}{c}{
  时间 (s)}&  文件(KB)\\\midrule[1pt]
\endfirsthead
\multicolumn{7}{c}{续表~\thetable\hskip1em 实验数据}\\
\toprule[1.5pt]
 测试程序 & \multicolumn{1}{c}{正常运行} & \multicolumn{1}{c}{同步} & \multicolumn{1}{c}{检查点} & \multicolumn{1}{c}{卷回恢复}
& \multicolumn{1}{c}{进程迁移} & \multicolumn{1}{c}{检查点} \\
& \multicolumn{1}{c}{时间 (s)}& \multicolumn{1}{c}{时间 (s)}&
\multicolumn{1}{c}{时间 (s)}& \multicolumn{1}{c}{时间 (s)}& \multicolumn{1}{c}{
  时间 (s)}&  文件(KB)\\\midrule[1pt]
\endhead
\hline
\multicolumn{7}{r}{续下页}
\endfoot
\endlastfoot
CG.A.2 & 23.05 & 0.002 & 0.116 & 0.035 & 0.589 & 32491 \\
CG.A.4 & 15.06 & 0.003 & 0.067 & 0.021 & 0.351 & 18211 \\
CG.A.8 & 13.38 & 0.004 & 0.072 & 0.023 & 0.210 & 9890 \\
CG.B.2 & 867.45 & 0.002 & 0.864 & 0.232 & 3.256 & 228562 \\
CG.B.4 & 501.61 & 0.003 & 0.438 & 0.136 & 2.075 & 123862 \\
CG.B.8 & 384.65 & 0.004 & 0.457 & 0.108 & 1.235 & 63777 \\
MG.A.2 & 112.27 & 0.002 & 0.846 & 0.237 & 3.930 & 236473 \\
MG.A.4 & 59.84 & 0.003 & 0.442 & 0.128 & 2.070 & 123875 \\
MG.A.8 & 31.38 & 0.003 & 0.476 & 0.114 & 1.041 & 60627 \\
MG.B.2 & 526.28 & 0.002 & 0.821 & 0.238 & 4.176 & 236635 \\
MG.B.4 & 280.11 & 0.003 & 0.432 & 0.130 & 1.706 & 123793 \\
MG.B.8 & 148.29 & 0.003 & 0.442 & 0.116 & 0.893 & 60600 \\
LU.A.2 & 2116.54 & 0.002 & 0.110 & 0.030 & 0.532 & 28754 \\
LU.A.4 & 1102.50 & 0.002 & 0.069 & 0.017 & 0.255 & 14915 \\
LU.A.8 & 574.47 & 0.003 & 0.067 & 0.016 & 0.192 & 8655 \\
LU.B.2 & 9712.87 & 0.002 & 0.357 & 0.104 & 1.734 & 101975 \\
LU.B.4 & 4757.80 & 0.003 & 0.190 & 0.056 & 0.808 & 53522 \\
LU.B.8 & 2444.05 & 0.004 & 0.222 & 0.057 & 0.548 & 30134 \\
EP.A.2 & 123.81 & 0.002 & 0.010 & 0.003 & 0.074 & 1834 \\
EP.A.4 & 61.92 & 0.003 & 0.011 & 0.004 & 0.073 & 1743 \\
EP.A.8 & 31.06 & 0.004 & 0.017 & 0.005 & 0.073 & 1661 \\
EP.B.2 & 495.49 & 0.001 & 0.009 & 0.003 & 0.196 & 2011 \\
EP.B.4 & 247.69 & 0.002 & 0.012 & 0.004 & 0.122 & 1663 \\
EP.B.8 & 126.74 & 0.003 & 0.017 & 0.005 & 0.083 & 1656 \\
\bottomrule[1.5pt]
\end{longtable}

\subsection{其它}
\label{sec:tableother}
如果不想让某个表格或者图片出现在索引里面,请使用命令 \cs{caption*}。
这个命令不会给表格编号,也就是出来的只有标题文字而没有“表~XX”,“图~XX”,否则
索引里面序号不连续就显得不伦不类,这也是 \LaTeX{} 里星号命令默认的规则。

有这种需求的多是本科同学的英文资料翻译部分,如果觉得附录中英文原文中的表格和图
片显示成“表”和“图”不协调的话,一个很好的办法就是用 \cs{caption*},参数
随便自己写,比如不守规矩的表~1.111 和图~1.111 能满足这种特殊需要(可以参看附录部
分)。
\begin{table}[ht]
  \begin{minipage}{0.4\linewidth}
    \centering
    \caption*{表~1.111\quad 这是一个手动编号,不出现在索引中的表格。}
    \label{tab:badtabular}
      \framebox(150,50)[c]{NJUrepo}
  \end{minipage}%
  \hfill%
  \begin{minipage}{0.4\linewidth}
    \centering
    \caption*{Figure~1.111\quad 这是一个手动编号,不出现在索引中的图。}
    \label{tab:badfigure}
    \framebox(150,50)[c]{薛瑞尼}
  \end{minipage}
\end{table}

如果的确想让它编号,但又不想让它出现在索引中的话,目前模板上不支持。

最后,虽然大家不一定会独立使用小页,但是关于小页中的脚注还是有必要提一下。请看下
面的例子。

\begin{minipage}[t]{\linewidth-2\parindent}
  柳宗元,字子厚(773-819),河东(今永济县)人\footnote{山西永济水饺。},是唐代
  杰出的文学家,哲学家,同时也是一位政治改革家。与韩愈共同倡导唐代古文运动,并称
  韩柳\footnote{唐宋八大家之首二位。}。
\end{minipage}

唐朝安史之乱后,宦官专权,藩镇割据,土地兼并日渐严重,社会生产破坏严重,民不聊生。柳宗
元对这种社会现实极为不满,他积极参加了王叔文领导的“永济革新”,并成为这一
运动的中坚人物。他们革除弊政,打击权奸,触犯了宦官和官僚贵族利益,在他们的联合反
扑下,改革失败了,柳宗元被贬为永州司马。

\section{定理环境}
\label{sec:theorem}

给大家演示一下各种和证明有关的环境:

\begin{assumption}
待月西厢下,迎风户半开;隔墙花影动,疑是玉人来。
\begin{align}
  \label{eq:eqnxmp}
  c & = a^2 - b^2 \\
    & = (a+b)(a-b)
\end{align}
\end{assumption}

千辛万苦,历尽艰难,得有今日。然相从数千里,未曾哀戚。今将渡江,方图百年欢笑,如
何反起悲伤?(引自《杜十娘怒沉百宝箱》)

\begin{definition}
子曰:「道千乘之国,敬事而信,节用而爱人,使民以时。」
\end{definition}

千古第一定义!问世间、情为何物,只教生死相许?天南地北双飞客,老翅几回寒暑。欢乐趣,离别苦,就中更有痴儿女。
君应有语,渺万里层云,千山暮雪,只影向谁去?

横汾路,寂寞当年箫鼓,荒烟依旧平楚。招魂楚些何嗟及,山鬼暗谛风雨。天也妒,未信与,莺儿燕子俱黄土。
千秋万古,为留待骚人,狂歌痛饮,来访雁丘处。

\begin{proposition}
 曾子曰:「吾日三省吾身 —— 为人谋而不忠乎?与朋友交而不信乎?传不习乎?」
\end{proposition}

多么凄美的命题啊!其日牛马嘶,新妇入青庐,奄奄黄昏后,寂寂人定初,我命绝今日,
魂去尸长留,揽裙脱丝履,举身赴清池,府吏闻此事,心知长别离,徘徊庭树下,自挂东南
枝。

\begin{remark}
天不言自高,水不言自流。
\begin{gather*}
\begin{split}
\varphi(x,z)
&=z-\gamma_{10}x-\gamma_{mn}x^mz^n\\
&=z-Mr^{-1}x-Mr^{-(m+n)}x^mz^n
\end{split}\\[6pt]
\begin{align} \zeta^0&=(\xi^0)^2,\\
\zeta^1 &=\xi^0\xi^1,\\
\zeta^2 &=(\xi^1)^2,
\end{align}
\end{gather*}
\end{remark}

天尊地卑,乾坤定矣。卑高以陈,贵贱位矣。 动静有常,刚柔断矣。方以类聚,物以群分,
吉凶生矣。在天成象,在地成形,变化见矣。鼓之以雷霆,润之以风雨,日月运行,一寒一
暑,乾道成男,坤道成女。乾知大始,坤作成物。乾以易知,坤以简能。易则易知,简则易
从。易知则有亲,易从则有功。有亲则可久,有功则可大。可久则贤人之德,可大则贤人之
业。易简,而天下矣之理矣;天下之理得,而成位乎其中矣。

\begin{axiom}
两点间直线段距离最短。
\begin{align}
x&\equiv y+1\pmod{m^2}\\
x&\equiv y+1\mod{m^2}\\
x&\equiv y+1\pod{m^2}
\end{align}
\end{axiom}

《彖曰》:大哉乾元,万物资始,乃统天。云行雨施,品物流形。大明始终,六位时成,时
乘六龙以御天。乾道变化,各正性命,保合大和,乃利贞。首出庶物,万国咸宁。

《象曰》:天行健,君子以自强不息。潜龙勿用,阳在下也。见龙再田,德施普也。终日乾
乾,反复道也。或跃在渊,进无咎也。飞龙在天,大人造也。亢龙有悔,盈不可久也。用九,
天德不可为首也。   

\newcommand\dif{\mathop{}\!\mathrm{d}}
\begin{lemma}
《猫和老鼠》是我最爱看的动画片。
\begin{multline*}%\tag*{[a]} % 这个不出现在索引中
\int_a^b\biggl\{\int_a^b[f(x)^2g(y)^2+f(y)^2g(x)^2]
 -2f(x)g(x)f(y)g(y)\dif x\biggr\}\dif y \\
 =\int_a^b\biggl\{g(y)^2\int_a^bf^2+f(y)^2
  \int_a^b g^2-2f(y)g(y)\int_a^b fg\biggr\}\dif y
\end{multline*}
\end{lemma}

行行重行行,与君生别离。相去万余里,各在天一涯。道路阻且长,会面安可知。胡马依北
风,越鸟巢南枝。相去日已远,衣带日已缓。浮云蔽白日,游子不顾返。思君令人老,岁月
忽已晚。  弃捐勿复道,努力加餐饭。

\begin{theorem}\label{the:theorem1}
犯我强汉者,虽远必诛\hfill —— 陈汤(汉)
\end{theorem}
\begin{subequations}
\begin{align}
y & = 1 \\
y & = 0
\end{align}
\end{subequations}
道可道,非常道。名可名,非常名。无名天地之始;有名万物之母。故常无,欲以观其妙;
常有,欲以观其徼。此两者,同出而异名,同谓之玄。玄之又玄,众妙之门。上善若水。水
善利万物而不争,处众人之所恶,故几于道。曲则全,枉则直,洼则盈,敝则新,少则多,
多则惑。人法地,地法天,天法道,道法自然。知人者智,自知者明。胜人者有力,自胜
者强。知足者富。强行者有志。不失其所者久。死而不亡者寿。

\begin{proof}
燕赵古称多感慨悲歌之士。董生举进士,连不得志于有司,怀抱利器,郁郁适兹土,吾
知其必有合也。董生勉乎哉?

夫以子之不遇时,苟慕义强仁者,皆爱惜焉,矧燕、赵之士出乎其性者哉!然吾尝闻
风俗与化移易,吾恶知其今不异于古所云邪?聊以吾子之行卜之也。董生勉乎哉?

吾因子有所感矣。为我吊望诸君之墓,而观于其市,复有昔时屠狗者乎?为我谢
曰:“明天子在上,可以出而仕矣!” \hfill —— 韩愈《送董邵南序》
\end{proof}

\begin{corollary}
  四川话配音的《猫和老鼠》是世界上最好看最好听最有趣的动画片。
\begin{alignat}{3}
V_i & =v_i - q_i v_j, & \qquad X_i & = x_i - q_i x_j,
 & \qquad U_i & = u_i,
 \qquad \text{for $i\ne j$;}\label{eq:B}\\
V_j & = v_j, & \qquad X_j & = x_j,
  & \qquad U_j & u_j + \sum_{i\ne j} q_i u_i.
\end{alignat}
\end{corollary}

迢迢牵牛星,皎皎河汉女。
纤纤擢素手,札札弄机杼。
终日不成章,泣涕零如雨。
河汉清且浅,相去复几许。
盈盈一水间,脉脉不得语。

\begin{example}
  大家来看这个例子。
\begin{equation}
\label{ktc}
\left\{\begin{array}{l}
\nabla f({\mbox{\boldmath $x$}}^*)-\sum\limits_{j=1}^p\lambda_j\nabla g_j({\mbox{\boldmath $x$}}^*)=0\\[0.3cm]
\lambda_jg_j({\mbox{\boldmath $x$}}^*)=0,\quad j=1,2,\cdots,p\\[0.2cm]
\lambda_j\ge 0,\quad j=1,2,\cdots,p.
\end{array}\right.
\end{equation}
\end{example}

\begin{exercise}
  请列出 Andrew S. Tanenbaum 和 W. Richard Stevens 的所有著作。
\end{exercise}

\begin{conjecture} \textit{Poincare Conjecture} If in a closed three-dimensional
  space, any closed curves can shrink to a point continuously, this space can be
  deformed to a sphere.
\end{conjecture}

\begin{problem}
 回答还是不回答,是个问题。
\end{problem}

如何引用定理~\ref{the:theorem1} 呢?加上 \cs{label} 使用 \cs{ref} 即可。妾发
初覆额,折花门前剧。郎骑竹马来,绕床弄青梅。同居长干里,两小无嫌猜。 十四为君妇,
羞颜未尝开。低头向暗壁,千唤不一回。十五始展眉,愿同尘与灰。常存抱柱信,岂上望夫
台。 十六君远行,瞿塘滟滪堆。五月不可触,猿声天上哀。门前迟行迹,一一生绿苔。苔深
不能扫,落叶秋风早。八月蝴蝶来,双飞西园草。感此伤妾心,坐愁红颜老。

\section{参考文献}
\label{sec:bib}
当然参考文献可以直接写 \cs{bibitem},虽然费点功夫,但是好控制,各种格式可以自己随意改
写。

本模板推荐使用 BIB\TeX,分别提供数字引用(\texttt{thuthesis-numeric.bst})和作
者年份引用(\texttt{thuthesis-author-year.bst})样式,基本符合学校的参考文献格式
(如专利等引用未加详细测试)。看看这个例子,关于书的~\cite{tex, companion,
  ColdSources},还有这些~\cite{Krasnogor2004e, clzs, zjsw},关于杂志
的~\cite{ELIDRISSI94, MELLINGER96, SHELL02},硕士论文~\cite{zhubajie,
  metamori2004},博士论文~\cite{shaheshang, FistSystem01},标准文
件~\cite{IEEE-1363},会议论文~\cite{DPMG,kocher99},技术报告~\cite{NPB2},电子文
献~\cite{chuban2001,oclc2000}。
若使用著者-出版年制,中文参考文献~\cite{cnarticle}应增加
\texttt{key=\{pinyin\}} 字段,以便正确进行排序~\cite{cnproceed}。
另外,如果对参考文献有不如意的地方,请手动修改 \texttt{bbl} 文件。

有时候不想要上标,那么可以这样~\inlinecite{shaheshang},这个非常重要。

有时候一些参考文献没有纸质出处,需要标注 URL。缺省情况下,URL 不会在连字符处断行,
这可能使得用连字符代替空格的网址分行很难看。如果需要,可以将模板类文件中
\begin{verbatim}
\RequirePackage{hyperref}
\end{verbatim}
一行改为:
\begin{verbatim}
\PassOptionsToPackage{hyphens}{url}
\RequirePackage{hyperref}
\end{verbatim}
使得连字符处可以断行。更多设置可以参考 \texttt{url} 宏包文档。

\section{公式}
\label{sec:equation}
\renewcommand\vec{\symbf}
\newcommand\mat{\symbf}
贝叶斯公式如式~(\ref{equ:chap1:bayes}),其中 $p(y|\vec{x})$ 为后验;
$p(\vec{x})$ 为先验;分母 $p(\vec{x})$ 为归一化因子。
\begin{equation}
\label{equ:chap1:bayes}
p(y|\vec{x}) = \frac{p(\vec{x},y)}{p(\vec{x})}=
\frac{p(\vec{x}|y)p(y)}{p(\vec{x})}
\end{equation}

论文里面公式越多,\TeX{} 就越 happy。再看一个 \pkg{amsmath} 的例子:
\newcommand{\envert}[1]{\left\lvert#1\right\rvert}
\begin{equation}\label{detK2}
\det\mat{K}(t=1,t_1,\dots,t_n)=\sum_{I\in\vec{n}}(-1)^{\envert{I}}
\prod_{i\in I}t_i\prod_{j\in I}(D_j+\lambda_jt_j)\det\vec{A}
^{(\lambda)}(\overline{I}|\overline{I})=0.
\end{equation}

前面定理示例部分列举了很多公式环境,可以说把常见的情况都覆盖了,大家在写公式的时
候一定要好好看 \pkg{amsmath} 的文档,并参考模板中的用法:
\begin{multline*}%\tag{[b]} % 这个出现在索引中的
\int_a^b\biggl\{\int_a^b[f(x)^2g(y)^2+f(y)^2g(x)^2]
 -2f(x)g(x)f(y)g(y)\dif x\biggr\}\dif y \\
 =\int_a^b\biggl\{g(y)^2\int_a^bf^2+f(y)^2
  \int_a^b g^2-2f(y)g(y)\int_a^b fg\biggr\}\dif y
\end{multline*}

其实还可以看看这个多级规划:
\begin{equation}\label{bilevel}
\left\{\begin{array}{l}
\max\limits_{{\mbox{\footnotesize\boldmath $x$}}} F(x,y_1^*,y_2^*,\cdots,y_m^*)\\[0.2cm]
\mbox{subject to:}\\[0.1cm]
\qquad G(x)\le 0\\[0.1cm]
\qquad(y_1^*,y_2^*,\cdots,y_m^*)\mbox{ solves problems }(i=1,2,\cdots,m)\\[0.1cm]
\qquad\left\{\begin{array}{l}
    \max\limits_{{\mbox{\footnotesize\boldmath $y_i$}}}f_i(x,y_1,y_2,\cdots,y_m)\\[0.2cm]
    \mbox{subject to:}\\[0.1cm]
    \qquad g_i(x,y_1,y_2,\cdots,y_m)\le 0.
    \end{array}\right.
\end{array}\right.
\end{equation}
这些跟规划相关的公式都来自于刘宝碇老师《不确定规划》的课件。

% \include{parts/examples/chap02}
% \include{parts/examples/digitalexp}
% \include{parts/examples/code}
% \backmatter
% \listoffigures
% \listoftables
% \listofequations
% \bibliographystyle{ref/numeric} % ref/numeric,ref/author-year,plainnat,IEEEtran
% \bibliography{ref/refs}
% % 如果使用声明扫描页,将可选参数指定为扫描后的 PDF 文件名,例如:
% \begin{acknowledgement}[scan-statement.pdf]
\begin{acknowledgement}
  衷心感谢导师 xxx 教授和物理系 xxx 副教授对本人的精心指导。他们的言传身教将使
  我终生受益。

  在美国麻省理工学院化学系进行九个月的合作研究期间,承蒙 xxx 教授热心指导与帮助,不
  胜感激。感谢 xx 实验室主任 xx 教授,以及实验室全体老师和同学们的热情帮助和支
  持!本课题承蒙国家自然科学基金资助,特此致谢。

  感谢 \LaTeX 和 njuthesis\cite{njuthesis},帮我节省了不少时间。
\end{acknowledgement}

% %% 附录
% \begin{appendix}
  % \chapter{外文资料原文}
\label{cha:engorg}

\title{The title of the English paper}

\textbf{Abstract:} As one of the most widely used techniques in operations
research, \emph{ mathematical programming} is defined as a means of maximizing a
quantity known as \emph{bjective function}, subject to a set of constraints
represented by equations and inequalities. Some known subtopics of mathematical
programming are linear programming, nonlinear programming, multiobjective
programming, goal programming, dynamic programming, and multilevel
programming$^{[1]}$.

It is impossible to cover in a single chapter every concept of mathematical
programming. This chapter introduces only the basic concepts and techniques of
mathematical programming such that readers gain an understanding of them
throughout the book$^{[2,3]}$.


\section{Single-Objective Programming}
The general form of single-objective programming (SOP) is written
as follows,
\begin{equation}\tag*{(123)} % 如果附录中的公式不想让它出现在公式索引中,那就请
                             % 用 \tag*{xxxx}
\left\{\begin{array}{l}
\max \,\,f(x)\\[0.1 cm]
\mbox{subject to:} \\ [0.1 cm]
\qquad g_j(x)\le 0,\quad j=1,2,\cdots,p
\end{array}\right.
\end{equation}
which maximizes a real-valued function $f$ of
$x=(x_1,x_2,\cdots,x_n)$ subject to a set of constraints.

\newcommand\Real{\mathbf{R}}
\newtheorem{mpdef}{Definition}[chapter]
\begin{mpdef}
In SOP, we call $x$ a decision vector, and
$x_1,x_2,\cdots,x_n$ decision variables. The function
$f$ is called the objective function. The set
\begin{equation}\tag*{(456)} % 这里同理,其它不再一一指定。
S=\left\{x\in\Real^n\bigm|g_j(x)\le 0,\,j=1,2,\cdots,p\right\}
\end{equation}
is called the feasible set. An element $x$ in $S$ is called a
feasible solution.
\end{mpdef}

\newtheorem{mpdefop}[mpdef]{Definition}
\begin{mpdefop}
A feasible solution $x^*$ is called the optimal
solution of SOP if and only if
\begin{equation}
f(x^*)\ge f(x)
\end{equation}
for any feasible solution $x$.
\end{mpdefop}

One of the outstanding contributions to mathematical programming was known as
the Kuhn-Tucker conditions\ref{eq:ktc}. In order to introduce them, let us give
some definitions. An inequality constraint $g_j(x)\le 0$ is said to be active at
a point $x^*$ if $g_j(x^*)=0$. A point $x^*$ satisfying $g_j(x^*)\le 0$ is said
to be regular if the gradient vectors $\nabla g_j(x)$ of all active constraints
are linearly independent.

Let $x^*$ be a regular point of the constraints of SOP and assume that all the
functions $f(x)$ and $g_j(x),j=1,2,\cdots,p$ are differentiable. If $x^*$ is a
local optimal solution, then there exist Lagrange multipliers
$\lambda_j,j=1,2,\cdots,p$ such that the following Kuhn-Tucker conditions hold,
\begin{equation}
\label{eq:ktc}
\left\{\begin{array}{l}
    \nabla f(x^*)-\sum\limits_{j=1}^p\lambda_j\nabla g_j(x^*)=0\\[0.3cm]
    \lambda_jg_j(x^*)=0,\quad j=1,2,\cdots,p\\[0.2cm]
    \lambda_j\ge 0,\quad j=1,2,\cdots,p.
\end{array}\right.
\end{equation}
If all the functions $f(x)$ and $g_j(x),j=1,2,\cdots,p$ are convex and
differentiable, and the point $x^*$ satisfies the Kuhn-Tucker conditions
(\ref{eq:ktc}), then it has been proved that the point $x^*$ is a global optimal
solution of SOP.

\subsection{Linear Programming}
\label{sec:lp}

If the functions $f(x),g_j(x),j=1,2,\cdots,p$ are all linear, then SOP is called
a {\em linear programming}.

The feasible set of linear is always convex. A point $x$ is called an extreme
point of convex set $S$ if $x\in S$ and $x$ cannot be expressed as a convex
combination of two points in $S$. It has been shown that the optimal solution to
linear programming corresponds to an extreme point of its feasible set provided
that the feasible set $S$ is bounded. This fact is the basis of the {\em simplex
  algorithm} which was developed by Dantzig as a very efficient method for
solving linear programming.
\begin{table}[ht]
\centering
  \centering
  \caption*{Table~1\hskip1em This is an example for manually numbered table, which
    would not appear in the list of tables}
  \label{tab:badtabular2}
  \begin{tabular}[c]{|m{1.5cm}|c|c|c|c|c|c|}\hline
    \multicolumn{2}{|c|}{Network Topology} & \# of nodes &
    \multicolumn{3}{c|}{\# of clients} & Server \\\hline
    GT-ITM & Waxman Transit-Stub & 600 &
    \multirow{2}{2em}{2\%}&
    \multirow{2}{2em}{10\%}&
    \multirow{2}{2em}{50\%}&
    \multirow{2}{1.2in}{Max. Connectivity}\\\cline{1-3}
    \multicolumn{2}{|c|}{Inet-2.1} & 6000 & & & &\\\hline
    \multirow{2}{1.5cm}{Xue} & Rui  & Ni &\multicolumn{4}{c|}{\multirow{2}*{\njuthesis}}\\\cline{2-3}
    & \multicolumn{2}{c|}{ABCDEF} &\multicolumn{4}{c|}{} \\\hline
\end{tabular}
\end{table}

Roughly speaking, the simplex algorithm examines only the extreme points of the
feasible set, rather than all feasible points. At first, the simplex algorithm
selects an extreme point as the initial point. The successive extreme point is
selected so as to improve the objective function value. The procedure is
repeated until no improvement in objective function value can be made. The last
extreme point is the optimal solution.

\subsection{Nonlinear Programming}

If at least one of the functions $f(x),g_j(x),j=1,2,\cdots,p$ is nonlinear, then
SOP is called a {\em nonlinear programming}.

A large number of classical optimization methods have been developed to treat
special-structural nonlinear programming based on the mathematical theory
concerned with analyzing the structure of problems.
\njuemblem{!}{3cm}
% \begin{figure}[h]
%   \centering
%   \includegraphics{njulogo0.pdf}
%   \caption*{Figure~1\quad This is an example for manually numbered figure,
%     which would not appear in the list of figures}
%   \label{tab:badfigure2}
% \end{figure}

Now we consider a nonlinear programming which is confronted solely with
maximizing a real-valued function with domain $\Real^n$.  Whether derivatives are
available or not, the usual strategy is first to select a point in $\Real^n$ which
is thought to be the most likely place where the maximum exists. If there is no
information available on which to base such a selection, a point is chosen at
random. From this first point an attempt is made to construct a sequence of
points, each of which yields an improved objective function value over its
predecessor. The next point to be added to the sequence is chosen by analyzing
the behavior of the function at the previous points. This construction continues
until some termination criterion is met. Methods based upon this strategy are
called {\em ascent methods}, which can be classified as {\em direct methods},
{\em gradient methods}, and {\em Hessian methods} according to the information
about the behavior of objective function $f$. Direct methods require only that
the function can be evaluated at each point. Gradient methods require the
evaluation of first derivatives of $f$. Hessian methods require the evaluation
of second derivatives. In fact, there is no superior method for all
problems. The efficiency of a method is very much dependent upon the objective
function.

\subsection{Integer Programming}

{\em Integer programming} is a special mathematical programming in which all of
the variables are assumed to be only integer values. When there are not only
integer variables but also conventional continuous variables, we call it {\em
  mixed integer programming}. If all the variables are assumed either 0 or 1,
then the problem is termed a {\em zero-one programming}. Although integer
programming can be solved by an {\em exhaustive enumeration} theoretically, it
is impractical to solve realistically sized integer programming problems. The
most successful algorithm so far found to solve integer programming is called
the {\em branch-and-bound enumeration} developed by Balas (1965) and Dakin
(1965). The other technique to integer programming is the {\em cutting plane
  method} developed by Gomory (1959).

\hfill\textit{Uncertain Programming\/}\quad(\textsl{BaoDing Liu, 2006.2})

\section*{References}
\noindent{\itshape NOTE: These references are only for demonstration. They are
  not real citations in the original text.}

\begin{translationbib}
\item Donald E. Knuth. The \TeX book. Addison-Wesley, 1984. ISBN: 0-201-13448-9
\item Paul W. Abrahams, Karl Berry and Kathryn A. Hargreaves. \TeX\ for the
  Impatient. Addison-Wesley, 1990. ISBN: 0-201-51375-7
\item David Salomon. The advanced \TeX book.  New York : Springer, 1995. ISBN:0-387-94556-3
\end{translationbib}

\chapter{外文资料的调研阅读报告或书面翻译}

\title{英文资料的中文标题}

{\heiti 摘要:} 本章为外文资料翻译内容。如果有摘要可以直接写上来,这部分好像没有
明确的规定。

\section{单目标规划}
北冥有鱼,其名为鲲。鲲之大,不知其几千里也。化而为鸟,其名为鹏。鹏之背,不知其几
千里也。怒而飞,其翼若垂天之云。是鸟也,海运则将徙于南冥。南冥者,天池也。
\begin{equation}\tag*{(123)}
 p(y|\mathbf{x}) = \frac{p(\mathbf{x},y)}{p(\mathbf{x})}=
\frac{p(\mathbf{x}|y)p(y)}{p(\mathbf{x})}
\end{equation}

吾生也有涯,而知也无涯。以有涯随无涯,殆已!已而为知者,殆而已矣!为善无近名,为
恶无近刑,缘督以为经,可以保身,可以全生,可以养亲,可以尽年。

\subsection{线性规划}
庖丁为文惠君解牛,手之所触,肩之所倚,足之所履,膝之所倚,砉然响然,奏刀騞然,莫
不中音,合于桑林之舞,乃中经首之会。
\begin{table}[ht]
\centering
  \centering
  \caption*{表~1\hskip1em 这是手动编号但不出现在索引中的一个表格例子}
  \label{tab:badtabular3}
  \begin{tabular}[c]{|m{1.5cm}|c|c|c|c|c|c|}\hline
    \multicolumn{2}{|c|}{Network Topology} & \# of nodes &
    \multicolumn{3}{c|}{\# of clients} & Server \\\hline
    GT-ITM & Waxman Transit-Stub & 600 &
    \multirow{2}{2em}{2\%}&
    \multirow{2}{2em}{10\%}&
    \multirow{2}{2em}{50\%}&
    \multirow{2}{1.2in}{Max. Connectivity}\\\cline{1-3}
    \multicolumn{2}{|c|}{Inet-2.1} & 6000 & & & &\\\hline
    \multirow{2}{1.5cm}{Xue} & Rui  & Ni &\multicolumn{4}{c|}{\multirow{2}*{\njuthesis}}\\\cline{2-3}
    & \multicolumn{2}{c|}{ABCDEF} &\multicolumn{4}{c|}{} \\\hline
\end{tabular}
\end{table}

文惠君曰:“嘻,善哉!技盖至此乎?”庖丁释刀对曰:“臣之所好者道也,进乎技矣。始臣之
解牛之时,所见无非全牛者;三年之后,未尝见全牛也;方今之时,臣以神遇而不以目视,
官知止而神欲行。依乎天理,批大郤,导大窾,因其固然。技经肯綮之未尝,而况大坬乎!
良庖岁更刀,割也;族庖月更刀,折也;今臣之刀十九年矣,所解数千牛矣,而刀刃若新发
于硎。彼节者有间而刀刃者无厚,以无厚入有间,恢恢乎其于游刃必有余地矣。是以十九年
而刀刃若新发于硎。虽然,每至于族,吾见其难为,怵然为戒,视为止,行为迟,动刀甚微,
謋然已解,如土委地。提刀而立,为之而四顾,为之踌躇满志,善刀而藏之。”

文惠君曰:“善哉!吾闻庖丁之言,得养生焉。”


\subsection{非线性规划}
孔子与柳下季为友,柳下季之弟名曰盗跖。盗跖从卒九千人,横行天下,侵暴诸侯。穴室枢
户,驱人牛马,取人妇女。贪得忘亲,不顾父母兄弟,不祭先祖。所过之邑,大国守城,小
国入保,万民苦之。孔子谓柳下季曰:“夫为人父者,必能诏其子;为人兄者,必能教其弟。
若父不能诏其子,兄不能教其弟,则无贵父子兄弟之亲矣。今先生,世之才士也,弟为盗
跖,为天下害,而弗能教也,丘窃为先生羞之。丘请为先生往说之。”
\njuemblem{!}{3cm}
% \begin{figure}[h]
%   \centering
%   \includegraphics{njulogo0.pdf}
%   \caption*{图~1\hskip1em 这是手动编号但不出现索引中的图片的例子}
%   \label{tab:badfigure3}
% \end{figure}

柳下季曰:“先生言为人父者必能诏其子,为人兄者必能教其弟,若子不听父之诏,弟不受
兄之教,虽今先生之辩,将奈之何哉?且跖之为人也,心如涌泉,意如飘风,强足以距敌,
辩足以饰非。顺其心则喜,逆其心则怒,易辱人以言。先生必无往。”

孔子不听,颜回为驭,子贡为右,往见盗跖。

\subsection{整数规划}
盗跖乃方休卒徒大山之阳,脍人肝而餔之。孔子下车而前,见谒者曰:“鲁人孔丘,闻将军
高义,敬再拜谒者。”谒者入通。盗跖闻之大怒,目如明星,发上指冠,曰:“此夫鲁国之
巧伪人孔丘非邪?为我告之:尔作言造语,妄称文、武,冠枝木之冠,带死牛之胁,多辞缪
说,不耕而食,不织而衣,摇唇鼓舌,擅生是非,以迷天下之主,使天下学士不反其本,妄
作孝弟,而侥幸于封侯富贵者也。子之罪大极重,疾走归!不然,我将以子肝益昼餔之膳。”


\chapter{其它附录}
前面两个附录主要是给本科生做例子。其它附录的内容可以放到这里,当然如果你愿意,可
以把这部分也放到独立的文件中,然后将其 \cs{input} 到主文件中。

% \end{appendix}
% \end{document}
% \end{latex}
%
% \subsection{选项}
% \label{sec:option}
% \DescribeOption{language}
% 论文的主要语言(默认:中文)。可选:\option{chinese},\option{english}。决定了封面、标题、定理的语言。
% \DescribeOption{open}
% 正规出版物的章节出现在奇数页,也就是右手边的页面,这就是 \option{right},。在这种情况下,如果前一章的最后一页也是奇数,那么模板会自动生成一个纯粹的空白页。
% 提交的作业如果是电子稿的话,可以使用连续页,即使用\option{any}
% \DescribeOption{wide}
% 是否使用宽页面。如果生成作业的话,宽页面或许好看。
% \DescribeOption{draft}
% 是否生成水印。生成的水印为 Draft 表示此文档尚为草稿
%
% \subsection{字体配置}
% \label{sec:font-config}
% 使用\CTeX\ 默认字体配置
% \subsubsection{字体命令}
% \label{sec:fontcmds}
% \myentry{字体}
% \DescribeMacro{\songti}
% \DescribeMacro{\fangsong}
% \DescribeMacro{\heiti}
% \DescribeMacro{\kaishu}
% 用来切换宋体、仿宋、黑体、楷体四种基本字体。
% \myentry{字号}
% \DescribeMacro{\chuhao}
% \DescribeMacro{\xiaochu}
% \DescribeMacro{\yihao}
% \DescribeMacro{\xiaoyi}
% \DescribeMacro{\bahao}
% 定义字体大小,分别为
% \begin{center}
% \begin{tabular}{llllll}
% \toprule
% \cs{chuhao} & \cs{xiaochu} & \cs{yihao}  & \cs{xiaoyi}    & \cs{erhao}  & \cs{xiaoer}\\
% \cs{sanhao} & \cs{xiaosan} & \cs{sihao}  & \cs{banxiaosi} & \cs{xiaosi} & \cs{dawu}\\
% \cs{wuhao}  & \cs{xiaowu}  & \cs{liuhao} & \cs{xiaoliu}   & \cs{qihao}  & \cs{bahao}\\\bottomrule
% \end{tabular}
% \end{center}
% 使用方法为:\cs{command}\oarg{num},其中 command 为字号命令,num 为行距。比
% 如 \cs{xiaosi}[1.5] 表示选择小四字体,行距 1.5 倍。写作指南要求表格中的字体
% 是 \cs{dawu},模板已经设置好了。
%
% \subsection{封面信息}
% 仿照parts/examples/cover.tex 进行设置
% \subsection{表格}
% \DescribeMacro{\figoptadd}
% \DescribeMacro{\figoptaddcap}
% 定义了两个简单的表格命令
% \begin{latex}
%  \figoptadd{option}{address}
%  \figoptaddcap{option}{address}{caption}
% \end{latex}
% \subsection{图片}
% \DescribeMacro{\tabncc}
% \DescribeMacro{\tabnc}
% 定义了两个简单的图片命令
% \begin{latex}
%  \tabncc{number-of-columns}{content}{caption}
%  \tabnc{number-of-columns}{content}
% \end{latex}
% \subsection{代码}
% \njurepo 预设了如下的代码环境
% \begin{longtable}{ccccc}
% \toprule
% code & cpseudo & cpp & shell & commandshell \\
% verilog & python & latex & &\\  
% \bottomrule
% \end{longtable}
% \subsection{文字}
% \begin{latex}
% \href{link}{words} # 插入链接
% \magenta{品红色字}
% \CJKunderline{下划线字}
% \end{latex}
% 更多关于预置宏包的内容,可见 README.md 以及 \ref{sec:loadpkg}
%
%
% \section{致谢}
% 感谢以下宏包的作者,本宏包从中使用了部分代码和借鉴:
% \begin{itemize}
%  \item 南京大学\njuthesis https://github.com/xueruini/njuthesis
% \end{itemize}
% 
% \StopEventually{}
%
% \section{实现细节}
% \subsection{基本信息}
%    \begin{macrocode}
%<*cls>
\NeedsTeXFormat{LaTeX2e}
\ProvidesClass{njurepo}[2019/01/25 1.0.0 Nanjing University Report Template]
%    \end{macrocode}
%
% \subsection{定义选项}
% \label{sec:defoption}
% 使用kvoptions宏包进行选项设置
%    \begin{macrocode}
\hyphenation{NJU-repo}
\def\njurepo{\textsc{NJU}\-\textsc{repo}}
\def\njuthesis{\textsc{Thu}\-\textsc{Thesis}}
\def\version{1.1.3}
\RequirePackage{kvoptions}
\SetupKeyvalOptions{
    family=nju,
    prefix=nju@,
    setkeys=\kvsetkeys
}
\DeclareStringOption[chinese]{language}[chinese]
\DeclareStringOption[any]{open}[any]
\DeclareBoolOption{wide}
\DeclareBoolOption{color}
\DeclareBoolOption{draft}
\DeclareDefaultOption{\PassOptionsToClass{\CurrentOption}{ctexbook}}

\ProcessKeyvalOptions*
%    \end{macrocode}
%
% 检测选项是否合法
%    \begin{macrocode}
\newcommand\nju@validate@key[1]{%
  \@ifundefined{nju@\csname nju@#1\endcsname true}{%
    \ClassError{njurepo}{Invalid value '\csname nju#1\endcsname'}{}
    }{%
      \csname nju@\csname nju@#1\endcsname true\endcsname
    }
}
\newif\ifnju@chinese
\newif\ifnju@english
\nju@validate@key{language}
\newif\ifnju@any
\newif\ifnju@right
\nju@validate@key{open}
%    \end{macrocode}
% 
% 使用ctexbook宏包
%    \begin{macrocode}
\LoadClass[a4paper,openany,UTF8,zihao=-4,scheme=plain]{ctexbook}
%    \end{macrocode}
%
% \subsection{加载宏包}
% \label{sec:loadpkg}
% 用于开发的宏包
%    \begin{macrocode}
\RequirePackage{etoolbox}
\RequirePackage{ifxetex}
\RequirePackage{xparse}
\RequirePackage{blindtext}
%    \end{macrocode}
% 用于图片的宏包
%    \begin{macrocode}
\RequirePackage{graphicx}
\graphicspath{{resources/logo/}}
\graphicspath{{resources/}}
\RequirePackage[labelformat=simple]{subcaption}
\RequirePackage{pdfpages}
\includepdfset{fitpaper=true}
\RequirePackage{tikz}
\usetikzlibrary{decorations.pathmorphing,graphs,calc}
\RequirePackage{njuvisual}
\RequirePackage{dirtree}
\RequirePackage{qtree}
%    \end{macrocode}
% 用于表格的宏包
%    \begin{macrocode}
\RequirePackage{array}
\RequirePackage{longtable}
\RequirePackage{booktabs}
\RequirePackage{multirow}
\RequirePackage{tabularx}
\RequirePackage{diagbox}
\RequirePackage{makecell}
\RequirePackage{float}
%    \end{macrocode}
% 用于符号或数学的宏包
%    \begin{macrocode}
\RequirePackage{CJKfntef}
\RequirePackage{amsmath}
\RequirePackage[amsmath, thmmarks, hyperref]{ntheorem}
\RequirePackage{physics}
\RequirePackage{latexsym}
\RequirePackage{bbding,stmaryrd}
\RequirePackage{siunitx}
%\RequirePackage{fontawesome}
%    \end{macrocode}
% 其它宏包
%    \begin{macrocode}
\RequirePackage[sort&compress]{natbib}
%    \end{macrocode}
%
% 超链接
%    \begin{macrocode}
\RequirePackage{hyperref}
\ifxetex
  \hypersetup{%
    CJKbookmarks=true}
\else
  \hypersetup{%
    unicode=true,
    CJKbookmarks=false}
\fi
\hypersetup{%
  linktoc=all,
  bookmarksnumbered=true,
  bookmarksopen=true,
  bookmarksopenlevel=1,
  breaklinks=true,
  colorlinks=true,
  linkcolor=blue,
  plainpages=false,
  pdfborder=0 0 0}	
\urlstyle{same}
\def\UrlBreaks{%
  \do\/%
  \do\a\do\b\do\c\do\d\do\e\do\f\do\g\do\h\do\i\do\j\do\k\do\l%
     \do\m\do\n\do\o\do\p\do\q\do\r\do\s\do\t\do\u\do\v\do\w\do\x\do\y\do\z%
  \do\A\do\B\do\C\do\D\do\E\do\F\do\G\do\H\do\I\do\J\do\K\do\L%
     \do\M\do\N\do\O\do\P\do\Q\do\R\do\S\do\T\do\U\do\V\do\W\do\X\do\Y\do\Z%
  \do0\do1\do2\do3\do4\do5\do6\do7\do8\do9\do=\do/\do.\do:%
  \do\*\do\-\do\~\do\'\do\"\do\-}
\Urlmuskip=0mu plus 0.1mu
%    \end{macrocode}
%
% 页眉页脚设置
%    \begin{macrocode}
\RequirePackage{fancyhdr}
\RequirePackage{notoccite}	
%    \end{macrocode}
%
% \subsection{页面设置}
% 使用了njuthesis的非本科生默认配置。
%    \begin{macrocode}
\RequirePackage{geometry}
\ifnju@wide 
\geometry{
    a4paper, %210*297mm
    hcentering,
    ignoreall,
    nomarginpar,
    left=10mm,
    headheight=5mm,
    headsep=5mm,
    textheight=237mm,
    bottom=29mm,
    footskip=6mm
}\else
\geometry{
    a4paper, %210*297mm
    hcentering,
    ignoreall,
    nomarginpar,
    left=30mm,
    headheight=5mm,
    headsep=5mm,
    textheight=237mm,
    bottom=29mm,
    footskip=6mm
}
\fi
%    \end{macrocode}
%
% \subsection{主文档格式}
% \label{sec:mainbody}
%
% \begin{macro}{\cleardoublepage}
%    \begin{macrocode}
\let\nju@cleardoublepage\cleardoublepage
\newcommand{\nju@clearemptydoublepage}{%
  \clearpage{\pagestyle{nju@empty}\nju@cleardoublepage}}
\let\cleardoublepage\nju@clearemptydoublepage
%    \end{macrocode}
% \end{macro}
%
% \begin{macro}{\frontmatter}
% \begin{macro}{\mainmatter}
% \begin{macro}{\backmatter}
%    \begin{macrocode}
\renewcommand\frontmatter{%
    \ifnju@right\cleardoublepage\else\clearpage\fi
    \@mainmatterfalse
    \pagenumbering{Roman}
    \pagestyle{nju@empty}}
\renewcommand\mainmatter{%
    \ifnju@right\cleardoublepage\else\clearpage\fi
    \@mainmattertrue
    \pagenumbering{arabic}
    \pagestyle{nju@headings}}
\renewcommand\backmatter{%
    \ifnju@right\cleardoublepage\else\clearpage\fi
    \@mainmattertrue}
%    \end{macrocode}
% \end{macro}
% \end{macro}
% \end{macro}
%
% \subsection{字体与字号}
% \label{sec:font}
% \subsubsection{英文字体}
% 配置英文字体。
%    \begin{macrocode}
\newcommand\nju@fontset{\csname g__ctex_fontset_tl\endcsname}
\ifthenelse{\equal{\nju@fontset}{fandol}}{
  \setmainfont[
    Extension      = .otf,
    UprightFont    = *-regular,
    BoldFont       = *-bold,
    ItalicFont     = *-italic,
    BoldItalicFont = *-bolditalic,
  ]{texgyretermes}
  \setsansfont[
    Extension      = .otf,
    UprightFont    = *-regular,
    BoldFont       = *-bold,
    ItalicFont     = *-italic,
    BoldItalicFont = *-bolditalic,
  ]{texgyreheros}
  \setmonofont[
    Extension      = .otf,
    UprightFont    = *-regular,
    BoldFont       = *-bold,
    ItalicFont     = *-italic,
    BoldItalicFont = *-bolditalic,
    Scale          = MatchLowercase,
  ]{texgyrecursor}
}{
  \setmainfont{Times New Roman}
  \setsansfont{Arial}
  \ifthenelse{\equal{\nju@fontset}{mac}}{
    \setmonofont[Scale=MatchLowercase]{Menlo}
  }{
    \setmonofont[Scale=MatchLowercase]{Courier New}
  }
}
%    \end{macrocode}
%
% \subsubsection{数学环境字体}
% 配置数学字体(使用unicode-math)
%    \begin{macrocode}
\RequirePackage{unicode-math}
\unimathsetup{
  math-style = ISO,
  bold-style = ISO,
  nabla      = upright,
  partial    = upright,
}
\IfFontExistsTF{STIX2Math.otf}{
  \setmathfont[StylisticSet=8]{STIX2Math.otf}
  \setmathfont[range={scr,bfscr},StylisticSet=1]{STIX2Math.otf}
  \IfFontExistsTF{XITSMath-Regular.otf}{
    \setmathfont[range={\partial,\lbrace,\rbrace}]{XITSMath-Regular.otf}
  }{
    \setmathfont[range={\partial,\lbrace,\rbrace}]{XITS Math}
  }
}{
  \setmathfont{XITS Math}
}
%    \end{macrocode}
%
% \subsubsection{数学环境符号}
% \begin{macro}{\ldots}
% 省略号一律居中,所以 \cs{ldots} 不再居于底部。
%    \begin{macrocode}
\ifnju@chinese
  \def\mathellipsis{\cdots}
\fi
%    \end{macrocode}
% \end{macro}
%
% \begin{macro}{\le}
% \begin{macro}{\ge}
% \begin{macro}{\leq}
% \begin{macro}{\geq}
% 小于等于号要使用倾斜的形式。
%    \begin{macrocode}
\protected\def\le{\leqslant}
\protected\def\ge{\geqslant}
\AtBeginDocument{%
  \renewcommand\leq{\leqslant}%
  \renewcommand\geq{\geqslant}%
}
%    \end{macrocode}
% \end{macro}
% \end{macro}
% \end{macro}
% \end{macro}
%
% \begin{macro}{\int}
% 积分号 \cs{int} 使用正体,并且上下限默认置于积分号上下两侧。
%    \begin{macrocode}
\removenolimits{%
  \int\iint\iiint\iiiint\oint\oiint\oiiint
  \intclockwise\varointclockwise\ointctrclockwise\sumint
  \intbar\intBar\fint\cirfnint\awint\rppolint
  \scpolint\npolint\pointint\sqint\intlarhk\intx
  \intcap\intcup\upint\lowint
}
%    \end{macrocode}
% \end{macro}
%
% \begin{macro}{\Re}
% \begin{macro}{\Im}
% \begin{macro}{\nabla}
% 实部、虚部操作符使用罗马体 $\mathrm{Re}$、$\mathrm{Im}$ 而不是 fraktur 体
% $\Re$、$\Im$。\cs{nabla} 使用粗正体。
%    \begin{macrocode}
\AtBeginDocument{%
  \renewcommand{\Re}{\operatorname{Re}}%
  \renewcommand{\Im}{\operatorname{Im}}%
  \renewcommand\nabla{\mbfnabla}%
}
%    \end{macrocode}
% \end{macro}
% \end{macro}
% \end{macro}
%
% \begin{macro}{\bm}
% \begin{macro}{\boldsymbol}
% 兼容旧的粗体命令:\pkg{bm} 的 \cs{bm} 和 \pkg{amsmath} 的 \cs{boldsymbol}。
%    \begin{macrocode}
\newcommand\bm{\symbf}
\renewcommand\boldsymbol{\symbf}
%    \end{macrocode}
% \end{macro}
% \end{macro}
%
% \begin{macro}{\square}
% 兼容 \pkg{amssymb} 中的命令。
%    \begin{macrocode}
\newcommand\square{\mdlgwhtsquare}
%    \end{macrocode}
% \end{macro}
%
% 允许太长的公式断行、分页等。
%    \begin{macrocode}
\allowdisplaybreaks[4]
\renewcommand\theequation{\ifnum \c@chapter>\z@ \thechapter-\fi\@arabic\c@equation}
%    \end{macrocode}
%
% 公式距前后文的距离由 4 个参数控制,参见 \cs{normalsize} 的定义。
%    \begin{macrocode}
\def\make@df@tag{\@ifstar\nju@make@df@tag@@\make@df@tag@@@}
\def\nju@make@df@tag@@#1{\gdef\df@tag{\nju@maketag{#1}\def\@currentlabel{#1}}}
\def\nju@maketag#1{\maketag@@@{(\ignorespaces #1\unskip\@@italiccorr)}}
\def\tagform@#1{\maketag@@@{(\ignorespaces #1\unskip\@@italiccorr)\equcaption{#1}}}
%    \end{macrocode}
% 修改 \cs{tagform} 会影响 \cs{eqref}。
%    \begin{macrocode}
\renewcommand{\eqref}[1]{\textup{(\ref{#1})}}
%    \end{macrocode}
%
% \subsubsection{中文字体}
% \pkg{ctex}在微软下使用雅黑字体,在macOS下使用苹方字体。这里不做更改。
%
% \subsubsection{字号}
% WORD 中的字号对应该关系如下(1bp = 72.27/72 pt):
% \begin{center}
% \begin{tabular}{llll}
% \toprule
% 初号 & 42bp & 14.82mm & 42.1575pt \\
% 小初 & 36bp & 12.70mm & 36.135 pt \\
% 一号 & 26bp & 9.17mm & 26.0975pt \\
% 小一 & 24bp & 8.47mm & 24.09pt \\
% 二号 & 22bp & 7.76mm & 22.0825pt \\
% 小二 & 18bp & 6.35mm & 18.0675pt \\
% 三号 & 16bp & 5.64mm & 16.06pt \\
% 小三 & 15bp & 5.29mm & 15.05625pt \\
% 四号 & 14bp & 4.94mm & 14.0525pt \\
% 小四 & 12bp & 4.23mm & 12.045pt \\
% 五号 & 10.5bp & 3.70mm & 10.59375pt \\
% 小五 & 9bp & 3.18mm & 9.03375pt \\
% 六号 & 7.5bp & 2.56mm & \\
% 小六 & 6.5bp & 2.29mm & \\
% 七号 & 5.5bp & 1.94mm & \\
% 八号 & 5bp & 1.76mm & \\\bottomrule
% \end{tabular}
% \end{center}
%
% \begin{macro}{\normalsize}
% 默认正文小四号 (12bp) 字,行距为固定值 20 bp。
%    \begin{macrocode}
\renewcommand\normalsize{%
  \@setfontsize\normalsize{12bp}{20bp}%
  \abovedisplayskip=12bp \@plus 2bp \@minus 2bp
  \abovedisplayshortskip=12bp \@plus 2bp \@minus 2bp
  \belowdisplayskip=\abovedisplayskip
  \belowdisplayshortskip=\abovedisplayshortskip}
%    \end{macrocode}
% \end{macro}
%
% \begin{macro}{\nju@def@fontsize}
% 根据习惯定义字号。用法:
% \cs{nju@def@fontsize}\marg{字号名称}\marg{磅数}
%
% 避免了字号选择和行距的紧耦合。所有字号定义时为单倍行距,并提供选项指定行距倍数。
%    \begin{macrocode}
\def\nju@def@fontsize#1#2{%
  \expandafter\newcommand\csname #1\endcsname[1][1.3]{%
    \fontsize{#2}{##1\dimexpr #2}\selectfont}}
%    \end{macrocode}
% \end{macro}
%
% \begin{macro}{\chuhao}
% \begin{macro}{\xiaochu}
% \begin{macro}{\yihao}
% \begin{macro}{\xiaoyi}
% \begin{macro}{\erhao}
% \begin{macro}{\xiaoer}
% \begin{macro}{\sanhao}
% \begin{macro}{\xiaosan}
% \begin{macro}{\sihao}
% \begin{macro}{\banxiaosi}
% \begin{macro}{\xiaosi}
% \begin{macro}{\dawu}
% \begin{macro}{\wuhao}
% \begin{macro}{\xiaowu}
% \begin{macro}{\liuhao}
% \begin{macro}{\xiaoliu}
% \begin{macro}{\qihao}
% \begin{macro}{\bahao}
% 一组字号定义。
%    \begin{macrocode}
\nju@def@fontsize{chuhao}{42bp}
\nju@def@fontsize{xiaochu}{36bp}
\nju@def@fontsize{yihao}{26bp}
\nju@def@fontsize{xiaoyi}{24bp}
\nju@def@fontsize{erhao}{22bp}
\nju@def@fontsize{xiaoer}{18bp}
\nju@def@fontsize{sanhao}{16bp}
\nju@def@fontsize{xiaosan}{15bp}
\nju@def@fontsize{sihao}{14bp}
\nju@def@fontsize{banxiaosi}{13bp}
\nju@def@fontsize{xiaosi}{12bp}
\nju@def@fontsize{dawu}{11bp}
\nju@def@fontsize{wuhao}{10.5bp}
\nju@def@fontsize{xiaowu}{9bp}
\nju@def@fontsize{liuhao}{7.5bp}
\nju@def@fontsize{xiaoliu}{6.5bp}
\nju@def@fontsize{qihao}{5.5bp}
\nju@def@fontsize{bahao}{5bp}
%    \end{macrocode}
% \end{macro}
% \end{macro}
% \end{macro}
% \end{macro}
% \end{macro}
% \end{macro}
% \end{macro}
% \end{macro}
% \end{macro}
% \end{macro}
% \end{macro}
% \end{macro}
% \end{macro}
% \end{macro}
% \end{macro}
% \end{macro}
% \end{macro}
% \end{macro}
%
%
% \subsubsection{中文标点}
%
% \newcommand\unicodechar[1]{U+#1(\symbol{"#1})}
% 由于 Unicode 的一些标点符号是中西文混用的:
% \unicodechar{00B7}、
% \unicodechar{2013}、
% \unicodechar{2014}、
% \unicodechar{2018}、
% \unicodechar{2019}、
% \unicodechar{201C}、
% \unicodechar{201D}、
% \unicodechar{2025}、
% \unicodechar{2026}、
% \unicodechar{2E3A},
% 所以要根据语言设置正确的字体。
% \footnote{\url{https://github.com/CTeX-org/ctex-kit/issues/389}}
% 所以要根据语言设置正确的字体。
%    \begin{macrocode}
\newcommand\nju@setchinese{%
  \xeCJKResetPunctClass
}
\newcommand\nju@setenglish{%
  \xeCJKDeclareCharClass{HalfLeft}{"2018, "201C}%
  \xeCJKDeclareCharClass{HalfRight}{
    "00B7, "2019, "201D, "2013, "2014, "2025, "2026, "2E3A,
  }%
}
\newcommand\nju@setdefaultlanguage{%
  \ifnju@chinese
    \nju@setchinese
  \else
    \nju@setenglish
  \fi
}
%    \end{macrocode}
%
% \subsection{局部设置}
% \subsubsection{页眉页脚}
% \label{sec:headerfooter}
%
% 定义页眉和页脚样式。
% \begin{macro}{\ps@nju@empty}
% \begin{macro}{\ps@nju@plain}
% \begin{macro}{\ps@nju@headings}
% \begin{itemize}
% \item \texttt{nju@empty}:页眉页脚都没有
% \item \texttt{nju@plain}:只显示页脚的页码。\cs{chapter} 自动调用
% \cs{thispagestyle\{nju@plain\}}。
% \item \texttt{nju@headings}:页眉页脚同时显示
% \end{itemize}
%    \begin{macrocode}
\fancypagestyle{nju@empty}{%
  \fancyhf{}
  \renewcommand{\headrulewidth}{0pt}
  \renewcommand{\footrulewidth}{0pt}}
\fancypagestyle{nju@plain}{%
  \fancyhead{}
  \fancyfoot[C]{\xiaowu\thepage}
  \renewcommand{\headrulewidth}{0pt}
  \renewcommand{\footrulewidth}{0pt}}
\fancypagestyle{nju@headings}{%
  \fancyhead{}
  \fancyhead[C]{\wuhao\normalfont\leftmark}
  \fancyfoot{}
  \fancyfoot[C]{\wuhao\thepage}
  \renewcommand{\headrulewidth}{0.4pt}
  \renewcommand{\footrulewidth}{0pt}}
%    \end{macrocode}
% \end{macro}
% \end{macro}
% \end{macro}
%
% \subsubsection{段落}
% \label{sec:paragraph}
%
% 全文首行缩进 2 字符,标点符号用全角
%    \begin{macrocode}
\ctexset{%
  punct=quanjiao,
  space=auto,
  autoindent=true}
%    \end{macrocode}
%
% \subsubsection{列表}
% 利用 \pkg{enumitem} 命令调整默认列表环境间的距离,以符合中文习惯。
%    \begin{macrocode}
\RequirePackage[shortlabels]{enumitem}
\RequirePackage{environ}
\setlist{nosep}
%    \end{macrocode}
%
%
% \subsubsection{脚注}
% \label{sec:footnote}
% 脚注符合中文习惯,数字带圈。
%    \begin{macrocode}
\ifthenelse{\equal{\nju@fontset}{mac}}{
  \newfontfamily\nju@circlefont{Songti SC Light}
}{
  \ifthenelse{\equal{\nju@fontset}{windows}}{
    \newfontfamily\nju@circlefont{SimSun}
  }{
    \IfFontExistsTF{XITS-Regular.otf}{
      \newfontfamily\nju@circlefont{XITS-Regular.otf}
    }{
      \newfontfamily\nju@circlefont{xits-regular.otf}
    }
  }
}
\def\nju@textcircled#1{%
  \ifnum\value{#1} >9%
    \ClassError{njurepo}%
      {Too many footnotes in this page.}{Keep footnote less than 10.}%
  \fi
  {\nju@circlefont\symbol{\numexpr\value{#1}+"245F\relax}}%
}
\renewcommand{\thefootnote}{\nju@textcircled{footnote}}
\renewcommand{\thempfootnote}{\nju@textcircled{mpfootnote}}
%    \end{macrocode}
%
% 定义脚注分割线,字号(宋体小五),以及悬挂缩进(1.5字符)。
%    \begin{macrocode}
\def\footnoterule{\vskip-3\p@\hrule\@width0.3\textwidth\@height0.4\p@\vskip2.6\p@}
\let\nju@footnotesize\footnotesize
\renewcommand\footnotesize{\nju@footnotesize\xiaowu[1.5]}
%\footnotemargin1.5em\relax
%    \end{macrocode}
%
% \cs{@makefnmark} 默认是上标样式,而在脚注部分要求为正文大小。利用\cs{patchcmd} 动态调整 \cs{@makefnmark} 的定义。
%    \begin{macrocode}
\let\nju@makefnmark\@makefnmark
\def\nju@@makefnmark{\hbox{{\normalfont\@thefnmark}}}
\pretocmd{\@makefntext}{\let\@makefnmark\nju@@makefnmark}{}{}
\apptocmd{\@makefntext}{\let\@makefnmark\nju@makefnmark}{}{}
%    \end{macrocode}
%
%
% \subsubsection{定理环境}
% \label{sec:equation}
%
% 定理标题使用黑体,正文使用宋体,冒号隔开。
%    \begin{macrocode}
\theorembodyfont{\normalfont}
\theoremheaderfont{\normalfont\heiti}
\theoremsymbol{\ensuremath{\square}}
\newtheorem*{proof}{证明}
\theoremsymbol{}
\theoremseparator{:}
\ifnju@chinese
  \theoremstyle{plain}
  \newcommand\nju@assumption@name{假设}
  \newcommand\nju@definition@name{定义}
  \newcommand\nju@proposition@name{命题}
  \newcommand\nju@lemma@name{引理}
  \newcommand\nju@theorem@name{定理}
  \newcommand\nju@axiom@name{公理}
  \newcommand\nju@corollary@name{推论}
  \newcommand\nju@exercise@name{练习}
  \newcommand\nju@example@name{例}
  \newcommand\nju@remark@name{注释}
  \newcommand\nju@conjecture@name{猜想}
  \theoremstyle{break}
  \newcommand\nju@problem@name{问题}
  \newcommand\nju@solution@name{解}
\else
  \theoremstyle{plain}
  \newcommand\nju@assumption@name{Assumption}
  \newcommand\nju@definition@name{Definition}
  \newcommand\nju@proposition@name{Proposition}
  \newcommand\nju@lemma@name{Lemma}
  \newcommand\nju@theorem@name{Theorem}
  \newcommand\nju@axiom@name{Axiom}
  \newcommand\nju@corollary@name{Corollary}
  \newcommand\nju@exercise@name{Exercise}
  \newcommand\nju@example@name{Example}
  \newcommand\nju@remark@name{Remark}
  \newcommand\nju@conjecture@name{Conjecture}
  \theoremstyle{break}
  \newcommand\nju@problem@name{Problem}
  \newcommand\nju@solution@name{Solution}
\fi
\theoremheaderfont{\bfseries}
\newtheorem{assumption}{\nju@assumption@name}[section]
\newtheorem{definition}{\nju@definition@name}[section]
\newtheorem{proposition}{\nju@proposition@name}[section]
\newtheorem{lemma}{\nju@lemma@name}[section]
\newtheorem{theorem}{\nju@theorem@name}[section]
\newtheorem{axiom}{\nju@axiom@name}[section]
\newtheorem{corollary}{\nju@corollary@name}[section]
\newtheorem{exercise}{\nju@exercise@name}[section]
\newtheorem{example}{\nju@example@name}[section]
\newtheorem{remark}{\nju@remark@name}[section]
\newtheorem{problem}{\nju@problem@name}[section]
\newtheorem{conjecture}{\nju@conjecture@name}[section]
\newtheorem{solution}{\nju@solution@name}[section]

%\RequirePackage{microtype}
\ifnju@chinese
\newcommand{\promisewords}{请独立完成作业,不得抄袭。\\若参考了其它资料,请给出引用。\\鼓励讨论,但需独立书写解题过程。}
\else
\newcommand{\promisewords}{I promise this work is done on my own with no plagiarism.}
\fi
\newcommand{\Hrule}{\noindent\rule{\linewidth}{0.5mm}}

\theorempostwork{\Hrule}
\newtheorem*{csolution}{\PencilRightDown\nju@solution@name}
\newtheorem*{nsolution}{\PencilRightDown\nju@solution@name}
\RequirePackage[listings]{tcolorbox}
\newtcolorbox{ps@problem}[1]{fonttitle=\bfseries,title=#1,before skip=0.5cm, after skip=-0.5cm}
\newenvironment{cproblem}[1][]{
    \begin{ps@problem}{\Checkmark \ \nju@problem@name\ #1}
}{
    \end{ps@problem}
}
\theoremstyle{plain}
\newtheorem*{nproblem}{\Checkmark \nju@problem@name}[section]
\newcommand{\pproblem}[1][1]{\begin{nproblem}[#1]\end{nproblem}}
% \subsubsection{浮动对象}
% \label{sec:float}
% 设置浮动对象和文字之间的距离
%    \begin{macrocode}
\setlength{\floatsep}{20bp \@plus4pt \@minus1pt}
\setlength{\intextsep}{20bp \@plus4pt \@minus2pt}
\setlength{\textfloatsep}{20bp \@plus4pt \@minus2pt}
\setlength{\@fptop}{0bp \@plus1.0fil}
\setlength{\@fpsep}{12bp \@plus2.0fil}
\setlength{\@fpbot}{0bp \@plus1.0fil}
%    \end{macrocode}
%
% 下面这组命令使浮动对象的缺省值稍微宽松一点,从而防止幅度对象占据过多的文本页面,
% 也可以防止在很大空白的浮动页上放置很小的图形。
%    \begin{macrocode}
\renewcommand{\textfraction}{0.15}
\renewcommand{\topfraction}{0.85}
\renewcommand{\bottomfraction}{0.65}
\renewcommand{\floatpagefraction}{0.60}
%    \end{macrocode}
%
% 定制浮动图形和表格标题样式
% \begin{itemize}
%   \item 图表标题字体为 11pt, 这里写作大五号
%   \item 去掉图表号后面的冒号。图序与图名文字之间空一个汉字符宽度。
%   \item 图:caption 在下,段前空 6 磅,段后空 12 磅
%   \item 表:caption 在上,段前空 12 磅,段后空 6 磅
% \end{itemize}
%
%    \begin{macrocode}
\let\old@tabular\@tabular
\def\nju@tabular{\dawu[1.5]\old@tabular}
\DeclareCaptionLabelFormat{nju}{{\dawu[1.5]\normalfont #1~#2}}
\DeclareCaptionLabelSeparator{nju}{\hspace{1em}}
\DeclareCaptionFont{nju}{\dawu[1.5]}
\captionsetup{labelformat=nju,labelsep=nju,font=nju,skip=6bp}
\captionsetup[table]{position=top}
\captionsetup[figure]{position=bottom}
\captionsetup[sub]{font=nju}
\renewcommand{\thesubfigure}{(\alph{subfigure})}
\renewcommand{\thesubtable}{(\alph{subtable})}
% \renewcommand{\p@subfigure}{:}
%    \end{macrocode}
% 我们采用 \pkg{longtable} 来处理跨页的表格。同样我们需要设置其默认字体为五号。
%    \begin{macrocode}
\let\nju@LT@array\LT@array
\def\LT@array{\dawu[1.5]\nju@LT@array} % set default font size
%    \end{macrocode}
%
% \begin{macro}{\hlinewd}
% 简单的表格使用三线表推荐用 \cs{hlinewd}。如果表格比较复杂还是用 \pkg{booktabs} 的命令好一些。
%    \begin{macrocode}
\def\hlinewd#1{%
  \noalign{\ifnum0=`}\fi\hrule \@height #1 \futurelet
    \reserved@a\@xhline}
%    \end{macrocode}
% \end{macro}
%
%
% \subsubsection{章节标题}
% \label{sec:theor}
%    \begin{macrocode}
\ifnju@chinese
  \ctexset{%
    chapter/name={第,章},
    appendixname=附录,
    contentsname={目\hspace{\ccwd}录},
    listfigurename=插图索引,
    listtablename=表格索引,
    figurename=图,
    tablename=表,
    bibname=参考文献,
    indexname=索引,
  }
  \newcommand\listequationname{公式索引}
  \newcommand\equationname{公式}
\else
  \newcommand\listequationname{List of Equations}
  \newcommand\equationname{Equation}
\fi
\newcommand{\cabstractname}{摘\hspace{\ccwd}要}
\newcommand{\eabstractname}{Abstract}
\let\CJK@todaysave=\today
\def\CJK@todaysmall@short{\the\year 年 \the\month 月}
\def\CJK@todaysmall{\the\year 年 \the\month 月 \the\day 日}
\def\CJK@todaybig@short{\zhdigits{\the\year}年\zhnumber{\the\month}月}
\def\CJK@todaybig{\zhdigits{\the\year}年\zhnumber{\the\month}月\zhnumber{\the\day}日}
\def\CJK@today{\CJK@todaysmall}
\renewcommand\today{\CJK@today}
\newcommand\CJKtoday[1][1]{%
  \ifcase#1\def\CJK@today{\CJK@todaysave}
    \or\def\CJK@today{\CJK@todaysmall}
    \or\def\CJK@today{\CJK@todaybig}
  \fi}
%    \end{macrocode}
%
% \pkg{fancyhdr} 定义页眉页脚很方便,但是有一个非常隐蔽的坑。通过 \pkg{fancyhdr}
% 定义的样式在第一次被调用时会修改 \cs{chaptermark},这会导致页眉信息错误(多余
% 章号并且英文大写)。这是因为在原始的 \file{book.cls} 中定义如下(大意):
% \begin{latex}
% \newcommand\chaptername{Chapter}
% \newcommand\@chapapp{\chaptername}
% \def\chaptermark#1{
%   \markboth{\MakeUppercase{\@chapapp\ \thechapter}}{}}
% \end{latex}
% 很显然这个 \cs{\@chapapp} 不适合中文,因此我们使用\cs{CTEXthechapter}(
% 如,“第 x 章”),同时会将 \cs{MakeUppercase} 去掉。也就是说我们会做如下动作:
% \begin{latex}
% \renewcommand{\chaptermark}[1]{\@mkboth{\CTEXthechapter\hskip\ccwd#1}{}}
% \end{latex}
% 但,\pkg{fancyhdr} 不知何故在 \cs{ps@fancy} 中对 \cs{chaptermark} 进行重定义
% (其实一模一样),而这个 \cs{ps@fancy} 会在 \cs{fancypagestyle} 中使用,如下:
% \begin{latex}
% \newcommand{\fancypagestyle}[2]{%
%   \@namedef{ps@#1}{\let\fancy@gbl\relax#2\relax\ps@fancy}}
% \end{latex}
% 这样的话,\cs{ps@fancy} 会在 \pkg{fancyhdr} 定义的任何样式首次样被激活时调用,从
% 而覆盖我们的 \cs{chaptermark} 定义(后续样式再激活不会重复覆盖)。所以我们采用如下
% 方法解决:
%    \begin{macrocode}
\AtBeginDocument{%
  \pagestyle{nju@empty}
  \renewcommand{\chaptermark}[1]{\@mkboth{\CTEXthechapter\hskip\ccwd#1}{}}}
%    \end{macrocode}
%
% 各级标题格式设置。
% \begin{description}
% \item[section] 章序号与章名之间空一个汉字符 黑体三号字,居中书写,单倍行距,段
%   前空 24 磅,段后空 18 磅。本科要求:段前段后间距 30/20 pt,行距 20pt。但正文
%   章节 30pt 的话和样例效果不一致。
% \item[section] 一级节标题,例如:\fbox{2.1 实验装置与实验方法}。节标题序号与标
%   题名之间空一个汉字符(下同)。采用黑体四号(14pt)字居左书写,行距为固定
%   值 20 磅,段前空 24 磅,段后空 6 磅。本科:25/12 pt,行距 18pt。
% \item[subsection] 二级节标题,例如:\fbox{2.1.1 实验装置}。采用黑体 13pt 字居左
%   书写,行距为固定值 20 磅,段前空 12 磅,段后空 6 磅。本科:中文黑体 12pt 字,
%   英文 13pt 字,段间距 12/6 pt,行距 15pt。
% \item[subsubsection] 三级节标题,例如:\fbox{2.1.2.1 归纳法}。采用黑体小四号
%   (12pt)字居左书写,行距为固定值 20 磅,段前空 12 磅,段后空 6 磅。
%
% \end{description}
%    \begin{macrocode}
\newcommand\nju@chapter@titleformat[1]{%
    \ifthenelse%
      {\equal{#1}{\eabstractname}}%
      {\bfseries #1}%
      {#1}%
  }
\ctexset{%
  chapter={
    afterindent=true,
    pagestyle={nju@headings},
    beforeskip={9bp},
    aftername=\hskip\ccwd,
    afterskip={24bp},
    format={\centering\sffamily\sanhao[1]},
    nameformat=\relax,
    numberformat=\relax,
    titleformat=\nju@chapter@titleformat,
    lofskip=0pt,
    lotskip=0pt,
  },
  section={
    afterindent=true,
    beforeskip={24bp\@plus 1ex \@minus .2ex},
    afterskip={6bp\@plus .2ex},
    format={\sffamily\sihao[1.429]},
  },
  subsection={
    afterindent=true,
    beforeskip={16bp\@plus 1ex \@minus .2ex},
    afterskip={6bp \@plus .2ex},
    format={\sffamily\banxiaosi[1.538]},
    numberformat={\sffamily\banxiaosi[1.538]},
  },
  subsubsection={
    afterindent=true,
    beforeskip={16bp\@plus 1ex \@minus .2ex},
    afterskip={6bp \@plus .2ex},
    format={\sffamily\xiaosi[1.667]},
  },
  paragraph/afterindent=true,
  subparagraph/afterindent=true}
%    \end{macrocode}
%
% \begin{macro}{\nju@chapter*}
% 默认的 \cs{chapter*} 很难同时满足研究生院和本科生的论文要求。本科论文要求所有的
% 章都出现在目录里,比如摘要、Abstract、主要符号表等,所以可以简单的扩展默
% 认\cs{chapter*} 实现这个目的。但是研究生又不要这些出现在目录中,而且致谢和声明
% 部分的章名、页眉和目录都不同,所以定义一个灵活的 \cs{nju@chapter*} 专门处理这些
% 要求。
%
% \cs{nju@chapter*}\oarg{tocline}\marg{title}\oarg{header}: tocline 是出现在目录
% 中的条目,如果为空则此 chapter 不出现在目录中,如果省略表示目录出现 title;
% title 是章标题;header 是页眉出现的标题,如果忽略则取 title。通过这个宏我才真
% 正体会到 \TeX\ macro 的力量!
%    \begin{macrocode}
\newcounter{nju@bookmark}
\NewDocumentCommand\nju@chapter{s o m o}{
  \IfBooleanF{#1}{%
    \ClassError{njurepo}{You have to use the star form: \string\nju@chapter*}{}
  }%
  \ifnju@right\cleardoublepage\else\clearpage\fi\phantomsection%
  \IfValueTF{#2}{%
    \ifthenelse{\equal{#2}{}}{%
      \addtocounter{nju@bookmark}\@ne
      \pdfbookmark[0]{#3}{njuchapter.\thenju@bookmark}
    }{%
      \addcontentsline{toc}{chapter}{#3}
    }
  }{%
    \addcontentsline{toc}{chapter}{#3}
  }%
  \ctexset{chapter/beforeskip=25bp}
  \chapter*{#3}%
  \ctexset{chapter/beforeskip=15bp}
  \IfValueTF{#4}{%
    \ifthenelse{\equal{#4}{}}
    {\@mkboth{}{}}
    {\@mkboth{#4}{#4}}
  }{%
    \@mkboth{#3}{#3}
  }
}
%    \end{macrocode}
% \end{macro}
%
%
% \subsubsection{目录}
% \label{sec:toc}
% 最多 4 层,即: x.x.x.x,对应的命令和层序号分别是:
% \cs{chapter}(0), \cs{section}(1), \cs{subsection}(2), \cs{subsubsection}(3)。
%    \begin{macrocode}
\setcounter{secnumdepth}{3}
\setcounter{tocdepth}{2}
%    \end{macrocode}
%
% 每章标题行前空 6 磅,后空 0 磅。章节名中英文用 Arial 字体,页码仍用 Times。
% \begin{macro}{\tableofcontents}
%    \begin{macrocode}
\renewcommand\tableofcontents{%
  \nju@chapter*[]{\contentsname}
  \xiaosi[1.65]\@starttoc{toc}\normalsize}
%    \end{macrocode}
% 调整目录样式
%    \begin{macrocode}
\def\@pnumwidth{2em}
\def\@tocrmarg{\@pnumwidth}
\def\@dotsep{1}
\renewcommand*\l@chapter[2]{%
  \ifnum \c@tocdepth >\m@ne
    \addpenalty{-\@highpenalty}%
    \vskip 4bp \@plus\p@
    \setlength\@tempdima{4em}%
    \begingroup
      \parindent \z@ \rightskip \@pnumwidth
      \parfillskip -\@pnumwidth
      \leavevmode
      \advance\leftskip\@tempdima
      \hskip -\leftskip
      {#1}%
      \leaders\hbox{$\m@th\mkern \@dotsep mu\hbox{.}\mkern \@dotsep mu$}\hfill%
      \nobreak{#2}\par
      \penalty\@highpenalty
    \endgroup
  \fi}

\patchcmd{\@dottedtocline}{\hb@xt@\@pnumwidth}{\hbox}{}{}
\renewcommand*\l@section{%
  \@dottedtocline{1}{\ccwd}{2.1em}}
\renewcommand*\l@subsection{%
  \@dottedtocline{2}{2\ccwd}{3em}}
\renewcommand*\l@subsubsection{%
  \@dottedtocline{3}{3.5em}{3.8em}}
%    \end{macrocode}
% \end{macro}
%
% \subsection{附加页面}
% \label{sec:etc}
%
% \subsubsection{封面}
% \label{sec:cover}
% 定义封面参数。
%    \begin{macrocode}
\def\nju@def@term#1{%
  \define@key{nju}{#1}{\csname #1\endcsname{##1}}
  \expandafter\gdef\csname #1\endcsname##1{%
    \expandafter\gdef\csname nju@#1\endcsname{##1}}
  \csname #1\endcsname{}}
\nju@def@term{ctitle}
\nju@def@term{csubtitle}
\nju@def@term{csubsubtitle}
\nju@def@term{etitle}
\nju@def@term{esubtitle}
\nju@def@term{esubsubtitle}
\nju@def@term{cauthor}
\nju@def@term{csupervisor}
\nju@def@term{cassosupervisor}
\nju@def@term{ccosupervisor}
\nju@def@term{eauthor}
\nju@def@term{esupervisor}
\nju@def@term{eassosupervisor}
\nju@def@term{ecosupervisor}
\nju@def@term{cdegree}
\nju@def@term{edegree}
\nju@def@term{cdepartment}
\nju@def@term{edepartment}
\nju@def@term{cmajor}
\nju@def@term{emajor}
\nju@def@term{cdate}
\nju@def@term{edate}
\nju@def@term{stdid}
\nju@def@term{mail}
\cdate{\CJK@todaybig@short}
\edate{\ifcase \month \or January\or February\or March\or April\or May%
       \or June\or July \or August\or September\or October\or November
       \or December\fi\unskip,\ \ \the\year}
%    \end{macrocode}
%
% \begin{environment}{cabstract}
% \begin{environment}{eabstract}
% 摘要最好以环境的形式出现(否则命令的形式会导致开始结束的括号距离太远,我不喜
% 欢),这就必须让环境能够自己保存内容留待以后使用。使用 \pkg{environ} 的
% \cs{Collect@Body} 来实现。
%    \begin{macrocode}
\newcommand{\nju@@cabstract}[1]{\long\gdef\nju@cabstract{#1}}
\newenvironment{cabstract}{\Collect@Body\nju@@cabstract}{}
\newcommand{\nju@@eabstract}[1]{\long\gdef\nju@eabstract{#1}}
\newenvironment{eabstract}{\Collect@Body\nju@@eabstract}{}
%    \end{macrocode}
% \end{environment}
% \end{environment}
%
% \begin{macro}{\nju@parse@keywords}
%   不同论文格式关键词之间的分割不太相同,我们用 \cs{ckeywords} 和
%    \cs{ekeywords} 来收集关键词列表,然后用本命令来生成符合要求的格式。
%    \begin{macrocode}
\def\nju@parse@keywords#1{
  \define@key{nju}{#1}{\csname #1\endcsname{##1}}
  \expandafter\gdef\csname nju@#1\endcsname{}
  \expandafter\gdef\csname #1\endcsname##1{
    \@for\reserved@a:=##1\do{
      \expandafter\ifx\csname nju@#1\endcsname\@empty\else
        \expandafter\g@addto@macro\csname nju@#1\endcsname{%
          \ignorespaces\csname nju@#1@separator\endcsname}
      \fi
      \expandafter\expandafter\expandafter\g@addto@macro%
        \expandafter\csname nju@#1\expandafter\endcsname\expandafter{\reserved@a}}}}
%    \end{macrocode}
% \end{macro}
% \begin{macro}{\ckeywords}
% \begin{macro}{\ekeywords}
% 利用 \cs{nju@parse@keywords} 来定义,内部通过 \cs{nju@ckeywords} 和
% \cs{nju@ekeywords} 来引用。
%    \begin{macrocode}
\nju@parse@keywords{ckeywords}
\nju@parse@keywords{ekeywords}
%    \end{macrocode}
% \end{macro}
% \end{macro}
%
% \begin{macro}{\njusetup}
% 由上可见,封面和封底有一大堆信息需要设置,为了简化操作界面,提供一
% 个 \cs{njusetup} 命令支持 key/value 的方式来设置。key 就是前面各个设置项的
% 名字。\note[说明:]{只能设置普通项,不支持环境项,
% 如 \texttt{cabstract} 和 \texttt{eabstract}。} 由于这些设置项被 \cs{makecover}
% 调用,所以此命令需要在 \cs{makecover} 之前被调用。
%    \begin{macrocode}
\def\njusetup{\kvsetkeys{nju}}
%    \end{macrocode}
% \end{macro}
%
% 定义封面用到的各种文字。
%    \begin{macrocode}
\def\nju@ckeywords@separator{;}
\def\nju@ekeywords@separator{;}
\def\nju@catalog@number@title{分类号}
\def\nju@id@title{编号}
\def\nju@title@sep{:}
\def\nju@schoolname{南京大学}
\def\nju@author@title{姓名}
\def\nju@department@title{系别}
\def\nju@major@title{专业}
\def\nju@supervisor@title{指导教师}
\def\nju@assosuper@title{辅导教师}
\def\nju@studentid@title{学号}
\def\nju@date@title{日期}
\def\nju@mail@title{邮箱}
\newcommand{\nju@ckeywords@title}{关键词:}
\def\nju@title@pre{}

\def\nju@eng@title@sep{:}
\def\nju@eng@author@title{Name}
\def\nju@eng@studentid@title{StdID}
\def\nju@eng@date@title{Date}
\def\nju@eng@mail@title{E-mail}
%    \end{macrocode}
%
% 中文小型标题
%    \begin{macrocode}
\renewcommand{\maketitle}{
  \nju@setup@pdfinfo
  \begin{center} {\LARGE \ifnju@chinese\nju@ctitle\else\nju@etitle\fi}
  \end{center}
  \hspace*{\fill}
  \ifnju@chinese
    \nju@author@title\nju@title@sep\CJKunderline{\nju@cauthor}
  \else
    \nju@eng@author@title\nju@eng@title@sep\underline{\nju@eauthor}
  \fi
  \hspace*{\fill}
  \ifx\nju@stdid\@empty\relax
  \else
    \ifnju@chinese
      \nju@studentid@title\nju@title@sep\CJKunderline{\nju@stdid}
    \else
      \nju@eng@studentid@title\nju@eng@title@sep\underline{\nju@stdid}
    \fi
  \fi
  \hspace*{\fill}
  \ifnju@chinese
    \nju@date@title\nju@title@sep\CJKunderline{\today}
  \else
    \nju@eng@date@title\nju@eng@title@sep\CJKunderline{\nju@edate}
  \fi
  \hspace*{\fill}\\
}
%    \end{macrocode}
%
% 别样封面
%    \begin{macrocode}
\newcommand{\maketitlepage}{
  \nju@setup@pdfinfo
  \begin{titlepage}
    \begin{center}
    \ifx\nju@esubsubtitle\@empty\relax  {\LARGE\sffamily\scshape\ifnju@chinese\nju@csubsubtitle\else\nju@esubsubtitle\fi\ }\\[1.5cm]
    \else
    {\LARGE\sffamily\scshape \ifnju@chinese\nju@csubsubtitle\else\nju@esubsubtitle\fi}\\[1.5cm]
    \fi
		{\Large\sffamily\scshape \ifnju@chinese\nju@csubtitle\else\nju@esubtitle\fi}\\
    \rule{\linewidth}{0.5mm} \\[0.4cm]
		{\huge\sffamily\bfseries \ifnju@chinese\nju@ctitle\else\nju@etitle\fi}\\
		\rule{\linewidth}{0.5mm} \\[1.5cm]
			
		\begin{center}
			\begin{tabular}{@{\hspace{0.5cm}}l@{\hspace{0.5cm}}l}
				\nju@eauthor & \nju@stdid\\
			\end{tabular}
		\end{center}
		\vfill
		{\large \nju@edate}
    \end{center}
    \ifnju@right\cleardoublepage\else\clearpage\fi
  \end{titlepage}
}
%    \end{macrocode}
%
% \myentry{封面第一页}
% \begin{macro}{\nju@first@titlepage}
% 题名使用一号黑体字,一行写不下时可分两行写,并采用 1.25 倍行距。
% 申请学位的学科门类: 小二号宋体字。
% 中文封面页边距:
%  上- 6.0 厘米,下- 5.5 厘米,左- 4.0 厘米,右- 4.0 厘米,装订线 0 厘米;
%
%    \begin{macrocode}
\newcommand\nju@underline[2][6em]{\hskip1pt\underline{\hb@xt@ #1{\hss#2\hss}}\hskip3pt}
\newlength{\nju@title@width}
\ifxetex % todo: ugly codes
  \newcommand{\nju@put@title}[2][\nju@title@width]{%
  \begin{CJKfilltwosides}[b]{#1}#2\end{CJKfilltwosides}}
\else
  \newcommand{\nju@put@title}[2][\nju@title@width]{%
  \begin{CJKfilltwosides}{#1}#2\end{CJKfilltwosides}}
\fi
\newcommand{\nju@first@titlepage}{
  \begin{center}
    \vspace*{-1.6cm}
    \parbox[b][2.4cm][t]{\textwidth}{%
      \rule{1cm}{0cm}}
      \vskip0.65cm
      {\includegraphics[width=0.3\textwidth]{njuname0.pdf}}
      \par\vskip1.5cm
      {\xiaochu\heiti\ziju{0.5}\textbf\nju@csubtitle}
      \vskip2.2cm\hskip0.8cm
      \noindent\heiti\xiaoer\nju@title@pre
      \parbox[t]{12cm}{%
      \ignorespaces\yihao[1.51]%
      \renewcommand{\CJKunderlinebasesep}{0.25cm}%
      \renewcommand{\ULthickness}{1.3pt}%
      \ifxetex
        \xeCJKsetup{underline/format=\color{black}}%
      \else
        \def\CJKunderlinecolor{\color{black}}%
      \fi
      \centering\CJKunderline*{\nju@ctitle}
      
    }%
      \vskip1.3cm
%    \end{macrocode}
%
% 作者及导师信息部分使用三号仿宋字
%    \begin{macrocode}
      \vskip0.75cm
      \ifx\nju@cassosupervisor\@empty%
        \def\nju@tempa{7.15cm}
      \else%
        \def\nju@tempa{8.15cm}
      \fi%
      \parbox[t][\nju@tempa][t]{\textwidth}{%
        {\fangsong\sanhao[1.95]%
         \hspace*{1.9cm}
         \setlength{\nju@title@width}{4em}
         \setlength{\extrarowheight}{6pt}
         \ifxetex % todo: ugly codes
           \begin{tabular}{p{\nju@title@width}@{}l@{\extracolsep{8pt}}l}
         \else
           \begin{tabular}{p{\nju@title@width}l@{}l}
         \fi
             \nju@put@title{\nju@department@title} & \nju@title@sep
               & \nju@cdepartment\\
             \nju@put@title{\nju@major@title}      & \nju@title@sep
               & \nju@cmajor\\
             \nju@put@title{\nju@author@title}     & \nju@title@sep
               & \nju@cauthor \\
             \nju@put@title{\nju@supervisor@title} & \nju@title@sep
               & \nju@csupervisor\\
             \ifx\nju@cassosupervisor\@empty\else%
               \nju@put@title{\nju@assosuper@title} & \nju@title@sep
               & \nju@cassosupervisor\\
             \fi
           \end{tabular}
        }}
%    \end{macrocode}
%
% 论文成文打印的日期,用三号宋体汉字,不用阿拉伯数字
% 本科:论文成文打印的日期用阿拉伯数字,采用小四号宋体
%    \begin{macrocode}
     \begin{center}
       {\vskip-1.0cm\xiaosi
         \songti\nju@cdate}
     \end{center}
    \end{center}} % end of titlepage
%    \end{macrocode}
% \end{macro}
%
% \myentry{英文封面}
% \begin{macro}{\nju@engcover}
%    \begin{macrocode}
\newcommand{\nju@engcover}{%
  \begin{center}
    \vspace*{-5pt}
    \parbox[t][5.2cm][t]{\paperwidth-7.2cm}{
      \renewcommand{\baselinestretch}{1.5}
      \begin{center}
        \erhao[1.1]\bfseries\sffamily\nju@etitle%
      \end{center}}
    \parbox[t][][b]{\paperwidth-7.2cm}{
      \renewcommand{\baselinestretch}{1.3}
      \begin{center}
        \sanhao\sffamily by\\[3bp]
        \bfseries\nju@eauthor%
        \ifx\nju@emajor\empty\relax\else
          \\(~\nju@emajor~)%
        \fi
      \end{center}}
    \par\vspace{0.9cm}
    \parbox[t][2.1cm][t]{\paperwidth-7.2cm}{
      \renewcommand{\baselinestretch}{1.2}
      \xiaosan\centering
      \begin{tabular}{rl}
        Supervisor : & \nju@esupervisor\\
        \ifx\nju@eassosupervisor\@empty
          \else Associate Supervisor : & \nju@eassosupervisor\\\fi
        \ifx\nju@ecosupervisor\@empty
          \else Cooperate Supervisor : & \nju@ecosupervisor\\\fi
      \end{tabular}}
    \parbox[t][2cm][b]{\paperwidth-7.2cm}{
    \begin{center}
      \sanhao\bfseries\sffamily\nju@edate
    \end{center}}
  \end{center}}
%    \end{macrocode}
% \end{macro}
%
% \begin{macro}{\makecover}
% 生成封面总命令。
%    \begin{macrocode}
\def\makecover{%
  \nju@setup@pdfinfo\nju@makecover}
\def\nju@setup@pdfinfo{%
  \ifnju@chinese
    \hypersetup{
      pdftitle    = \nju@ctitle,
      pdfauthor   = \nju@cauthor,
      pdfsubject  = \nju@cdegree,
      pdfkeywords = \nju@ckeywords,
    }%
  \else
    \hypersetup{
      pdftitle    = \nju@etitle,
      pdfauthor   = \nju@eauthor,
      pdfsubject  = \nju@edegree,
      pdfkeywords = \nju@ekeywords,
    }%
  \fi
  \hypersetup{
    pdfcreator={\njurepo-v\version}}}
\NewDocumentCommand{\nju@makecover}{o}{
  \phantomsection
  \pdfbookmark[-1]{\nju@ctitle}{ctitle}
  \normalsize%
  \begin{titlepage}
    \ifnju@chinese
      \nju@first@titlepage
    \else
      \nju@engcover
    \fi
    \ifnju@right\cleardoublepage\else\clearpage\fi
  \end{titlepage}
}
\newcommand{\makeabstract}{
  \normalsize
  \nju@makeabstract
  \let\@tabular\nju@tabular
}
%    \end{macrocode}
% \end{macro}
%
% \subsubsection{摘要}
% \label{sec:abstractformat}
%
% \begin{macro}{\nju@put@keywords}
% 排版关键字。
%    \begin{macrocode}
\newbox\nju@kw
\newcommand\nju@put@keywords[2]{%
  \begingroup
    \setbox\nju@kw=\hbox{#1}
    \indent%
    \box\nju@kw#2\par
  \endgroup}
%    \end{macrocode}
% \end{macro}
%
% \begin{macro}{\nju@makeabstract}
% 中文摘要部分的标题为“\textbf{摘要}”,用黑体三号字。
%    \begin{macrocode}
\newcommand{\nju@makeabstract}{%
  \clearpage
  \pagestyle{nju@plain}
  \pagenumbering{Roman}
%    \end{macrocode}
%
% 摘要内容用小四号字书写,两端对齐,汉字用宋体,外文字用 Times New Roman 体,
% 标点符号一律用中文输入状态下的标点符号。
%    \begin{macrocode}
  \ifnju@chinese
    \nju@setchinese
    \nju@chapter*[]{\cabstractname} % no tocline
    \nju@cabstract
    \vskip12bp
    \nju@put@keywords{\textbf\nju@ckeywords@title}{\nju@ckeywords}
  \else
  \nju@setenglish
    \nju@chapter*[]{\eabstractname} % no tocline
    \nju@eabstract
    \vskip12bp
    \nju@put@keywords{%
      \textbf{Key Words:\enskip}}{\nju@ekeywords}%
  \fi
  \nju@setdefaultlanguage
}
%    \end{macrocode}
% \end{macro}
%
%
%
% \subsubsection{主要符号表}
% \label{sec:denotationfmt}
% \begin{environment}{denotation}
% 主要符号对照表。
%    \begin{macrocode}
\ifnju@chinese
  \newcommand\nju@denotation@name{主要符号对照表}
\else
  \newcommand\nju@denotation@name{Nomenclature}
\fi
\newenvironment{denotation}[1][2.5cm]{%
  \nju@chapter*[]{\nju@denotation@name} % no tocline
  \vskip-30bp\xiaosi[1.6]\begin{nju@denotation}[labelwidth=#1]
}{%
  \end{nju@denotation}
}
\newlist{nju@denotation}{description}{1}
\setlist[nju@denotation]{%
  nosep,
  font=\normalfont,
  align=left,
  leftmargin=!, % sum of the following 3 lengths
  labelindent=0pt,
  labelwidth=2.5cm,
  labelsep*=0.5cm,
  itemindent=0pt,
}
%    \end{macrocode}
% \end{environment}
%
% \subsubsection{致谢与声明}
% \label{sec:ackanddeclare}
%
% \begin{environment}{acknowledgement}
% 支持扫描文件替换。
%    \begin{macrocode}
\ifnju@chinese
  \newcommand\nju@ack@name{致\hspace{\ccwd}谢}
\else
  \newcommand\nju@ack@name{Acknowledgments}
\fi
\newcommand\nju@declarename{声\hspace{\ccwd}明}
\newcommand{\nju@declaretext}{本人郑重声明:所呈交的学位论文,是本人在导师指导下
  ,独立进行研究工作所取得的成果。尽我所知,除文中已经注明引用的内容外,本学位论
  文的研究成果不包含任何他人享有著作权的内容。对本论文所涉及的研究工作做出贡献的
  其他个人和集体,均已在文中以明确方式标明。}
\newcommand{\nju@signature}{签\hspace{1em}名:}
\newcommand{\nju@backdate}{日\hspace{1em}期:}
%    \end{macrocode}
%
%  \cs{cleardoublepage},使新开章节的页码到达正确的状态。否则会因为 \cs{addcontentsline}
% 在 chapter 之前而导致目录页码错误。
% 定义致谢与声明环境。
%    \begin{macrocode}
\NewDocumentEnvironment{acknowledgement}{o}{%
    \nju@chapter*{\nju@ack@name}
  }
%    \end{macrocode}
%
% 声明部分
%    \begin{macrocode}
  {
    \ifnju@english\relax\else%
      \IfNoValueTF{#1}{%
        \nju@chapter*{\nju@declarename}
        \par{\xiaosi\parindent2em\nju@declaretext}\vskip2cm
        {\xiaosi\hfill\nju@signature\nju@underline[2.5cm]\relax%
         \nju@backdate\nju@underline[2.5cm]\relax}%
      }{%
        \includepdf[pagecommand={\thispagestyle{nju@empty}%
          \phantomsection\addcontentsline{toc}{chapter}{\nju@declarename}%
        }]{#1}%
      }%
    \fi
  }
%    \end{macrocode}
% \end{environment}
%
% \subsubsection{图表索引}
% \label{sec:threeindex}
% \begin{macro}{\listoffigures}
% \begin{macro}{\listoffigures*}
% \begin{macro}{\listoftables}
% \begin{macro}{\listoftables*}
% 定义图表以及公式目录样式。
%    \begin{macrocode}
\def\nju@starttoc#1{% #1: float type, prepend type name in \listof*** entry.
  \let\oldnumberline\numberline
  \def\numberline##1{\oldnumberline{\csname #1name\endcsname\hskip.4em ##1}}
  \@starttoc{\csname ext@#1\endcsname}
  \let\numberline\oldnumberline}
\def\nju@listof#1{% #1: float type
  \@ifstar
    {\nju@chapter*[]{\csname list#1name\endcsname}\nju@starttoc{#1}}
    {\nju@chapter*{\csname list#1name\endcsname}\nju@starttoc{#1}}}
\renewcommand\listoffigures{\nju@listof{figure}}
\renewcommand*\l@figure{\addvspace{6bp}\@dottedtocline{1}{0em}{4em}}
\renewcommand\listoftables{\nju@listof{table}}
\let\l@table\l@figure
%    \end{macrocode}
% \end{macro}
% \end{macro}
% \end{macro}
% \end{macro}
%
% \begin{macro}{\equcaption}
%   本命令只是为了生成公式列表,所以这个 caption 是假的。如果要编号最好用
%    equation 环境,如果是其它编号环境,请手动添加 \cs{equcaption}。
% 用法如下:
%
% \cs{equcaption}\marg{counter}
%
% \marg{counter} 指定出现在索引中的编号,一般取 \cs{theequation},如果你是用
%  \pkg{amsmath} 的 \cs{tag},那么默认是 \cs{tag} 的参数;除此之外可能需要你
% 手工指定。
%
%    \begin{macrocode}
\def\ext@equation{loe}
\def\equcaption#1{%
  \addcontentsline{\ext@equation}{equation}%
                  {\protect\numberline{#1}}}
%    \end{macrocode}
% \end{macro}
%
% \begin{macro}{\listofequations}
% \begin{macro}{\listofequations*}
% \LaTeX\ 默认没有公式索引,此处定义自己的 \cs{listofequations}。
%    \begin{macrocode}
\newcommand\listofequations{\nju@listof{equation}}
\let\l@equation\l@figure
%    \end{macrocode}
% \end{macro}
% \end{macro}
%
% \subsection{参考文献}
% \label{sec:ref}
%
% \begin{macro}{\inlinecite}
% 依赖于 \pkg{natbib} 宏包,修改其中的命令。 旧命令 \cs{onlinecite} 依然可用。
%    \begin{macrocode}
\newcommand\bibstyle@inline{\bibpunct{[}{]}{,}{n}{,}{,}}
\DeclareRobustCommand\inlinecite{\@inlinecite}
\def\@inlinecite#1{\begingroup\let\@cite\NAT@citenum\citep{#1}\endgroup}
\let\onlinecite\inlinecite
%    \end{macrocode}
% \end{macro}
%
% 参考文献的正文部分用五号字。
% 行距采用固定值 16 磅,段前空 3 磅,段后空 0 磅。
%
% 复用 \pkg{natbib} 的 \texttt{thebibliography} 环境,调整距离。
%    \begin{macrocode}
\renewcommand\bibsection{\nju@chapter*{\bibname}}
\renewcommand\bibfont{\wuhao[1.5]}
\setlength\bibhang{2\ccwd}
\addtolength{\bibsep}{-0.7em}
\setlength{\labelsep}{0.4em}
\def\@biblabel#1{[#1]\hfill}
%    \end{macrocode}
%
% 两种引用样式:
%    \begin{macrocode}
\expandafter\newcommand\csname bibstyle@numeric\endcsname{%
  \bibpunct{[}{]}{,}{s}{,}{\textsuperscript{,}}}
\expandafter\newcommand\csname bibstyle@author-year\endcsname{%
  \bibpunct{(}{)}{;}{a}{,}{,}}
%    \end{macrocode}
%
% 下面修改 \pkg{natbib} 的引用格式,主要是将页码写在上标位置。
% numeric 模式的 \cs{citet} 的页码:
%    \begin{macrocode}
\patchcmd\NAT@citexnum{%
  \@ifnum{\NAT@ctype=\z@}{%
    \if*#2*\else\NAT@cmt#2\fi
  }{}%
  \NAT@mbox{\NAT@@close}%
}{%
  \NAT@mbox{\NAT@@close}%
  \@ifnum{\NAT@ctype=\z@}{%
    \if*#2*\else\textsuperscript{#2}\fi
  }{}%
}{}{}
%    \end{macrocode}
%
% Numeric 模式的 \cs{citep} 的页码:
%    \begin{macrocode}
\renewcommand\NAT@citesuper[3]{\ifNAT@swa
  \if*#2*\else#2\NAT@spacechar\fi
\unskip\kern\p@\textsuperscript{\NAT@@open#1\NAT@@close\if*#3*\else#3\fi}%
   \else #1\fi\endgroup}
%    \end{macrocode}
%
% Author-year 模式的 \cs{citet} 的页码:
%    \begin{macrocode}
\patchcmd{\NAT@citex}{%
  \if*#2*\else\NAT@cmt#2\fi
  \if\relax\NAT@date\relax\else\NAT@@close\fi
}{%
  \if\relax\NAT@date\relax\else\NAT@@close\fi
  \if*#2*\else\textsuperscript{#2}\fi
}{}{}
%    \end{macrocode}
%
% Author-year 模式的 \cs{citep} 的页码:
%    \begin{macrocode}
\renewcommand\NAT@citesuper[3]{\ifNAT@swa
  \if*#2*\else#2\NAT@spacechar\fi
\unskip\kern\p@\textsuperscript{\NAT@@open#1\NAT@@close\if*#3*\else#3\fi}%
   \else #1\fi\endgroup}
%    \end{macrocode}
%
% 在顺序编码制下,\pkg{natbib} 只有在三个以上连续文献引用才会使用连接号,
% 这里修改为允许两个引用使用连接号。
%    \begin{macrocode}
\patchcmd{\NAT@citexnum}{%
  \ifx\NAT@last@yr\relax
    \def@NAT@last@yr{\@citea}%
  \else
    \def@NAT@last@yr{--\NAT@penalty}%
  \fi
}{%
  \def@NAT@last@yr{-\NAT@penalty}%
}{}{}
%    \end{macrocode}
%
% \subsection{附录}
% \label{sec:appendix}
% \begin{environment}{appendix}
% 主要给本科做外文翻译用。
%    \begin{macrocode}
\let\nju@appendix\appendix
\renewenvironment{appendix}{%
  \let\title\nju@appendix@title
  \nju@appendix}{%
  \let\title\@gobble}
%    \end{macrocode}
% \end{environment}
%
% \begin{macro}{\title}
% 本科外文翻译文章的标题,用法:\cs{title}\marg{资料标题}。这个命令只能在附录环
% 境下使用。
%    \begin{macrocode}
\let\title\@gobble
\newcommand{\nju@appendix@title}[1]{%
  \begin{center}
    \xiaosi[1.667] #1
  \end{center}}
%    \end{macrocode}
% \end{macro}
%
% \begin{environment}{translationbib}
% 外文资料的参考文用宋体五号字,取固定行距17pt,段前后3pt。
%    \begin{macrocode}
\newlist{translationbib}{enumerate}{1}
\setlist[translationbib]{label=[\arabic*],align=left,nosep,itemsep=6bp,
  leftmargin=10mm,labelsep=!,before=\vspace{0.5\baselineskip}\wuhao[1.3]}
%    \end{macrocode}
% \end{environment}
%\marginpar{这是边注}
%
%\subsection{颜色}
%    \begin{macrocode}
\RequirePackage{xcolor}
\definecolor{codegreen}{rgb}{0,0.6,0}
\definecolor{codegray}{rgb}{0.5,0.5,0.5}
\definecolor{codepurple}{rgb}{0.58,0,0.82}
\definecolor{backcolour}{rgb}{0.95,0.95,0.92}
\newcommand{\red}[1]{\textcolor{red}{#1}}
\newcommand{\redoverlay}[2]{\textcolor<#2>{red}{#1}}
\newcommand{\green}[1]{\textcolor{green}{#1}}
\newcommand{\greenoverlay}[2]{\textcolor<#2>{green}{#1}}
\newcommand{\blue}[1]{\textcolor{blue}{#1}}
\newcommand{\blueoverlay}[2]{\textcolor<#2>{blue}{#1}}
\newcommand{\purple}[1]{\textcolor{purple}{#1}}
\newcommand{\cyan}[1]{\textcolor{cyan}{#1}}
\newcommand{\teal}[1]{\textcolor{teal}{#1}}
\newcommand{\magenta}[1]{{\color{magenta}#1}}
\newcommand{\note}[2][Note]{{%
  \color{magenta}{\bfseries #1}\emph{#2}}}
%    \end{macrocode}
%
%\subsection{代码}
%    \begin{macrocode}
\RequirePackage{verbatim}
\RequirePackage{algorithm}
\RequirePackage[noend]{algpseudocode}
\newcommand{\pseduo}[2]{
\begin{algorithm}
	\caption{\textsc{#1}}
	\label{alg:#1}
	\begin{algorithmic}[1]
		#2
	\end{algorithmic}
\end{algorithm}
}
\RequirePackage{listings}
\lstdefinestyle{lstStyleBase}{%
   basicstyle=\small\ttfamily,
   aboveskip=\medskipamount,
   belowskip=\medskipamount,
   lineskip=0pt,
   boxpos=c,
   showlines=false,
   extendedchars=true,
   upquote=true,
   tabsize=2,
   showtabs=false,
   showspaces=false,
   showstringspaces=false,
   numbers=none,
   linewidth=\linewidth,
   xleftmargin=4pt,
   xrightmargin=0pt,
   resetmargins=false,
   breaklines=true,
   breakatwhitespace=false,
   breakindent=0pt,
   breakautoindent=true,
   columns=flexible,
   keepspaces=true,
   gobble=2,
   framesep=3pt,
   rulesep=1pt,
   framerule=1pt,
   backgroundcolor=\color{gray!5},
   stringstyle=\color{green!40!black!100},
   keywordstyle=\bfseries\color{blue!50!black},
   commentstyle=\slshape\color{black!60}
}

\newtcblisting{commandshell}{colback=black,colupper=white,colframe=yellow!75!black, listing only,listing options={style=tcblatex,language=sh},
every listing line={\textcolor{red}{\small\ttfamily\bfseries \$>}}}

\lstdefinestyle{lstStyleShell}{%
   style=lstStyleBase,
   frame=l,
   rulecolor=\color{purple},
   language=bash}

\lstdefinestyle{lstStyleLaTeX}{%
   style=lstStyleBase,
   frame=l,
   rulecolor=\color{violet},
   language=[LaTeX]TeX}

\lstdefinestyle{lstStylecdisplay}{%
  style=lstStyleBase,
  frame=tb,
  rulecolor=\color{cyan},
  keywordstyle=\color{magenta}\bfseries\ttfamily,
  commentstyle=\color{codegreen}\ttfamily,
	stringstyle=\color{codepurple}\ttfamily\sffamily,
	backgroundcolor=\color{backcolour},
	captionpos=b,
	numbers=left,
	numberstyle=\footnotesize\color{codegray},
	stepnumber=1,
  numbersep=5pt,
  language=C
}

\lstdefinestyle{lstStylecpseudo}{%
  style=lstStyleBase,
  frame=none,
  keywordstyle=\color{magenta}\bfseries\ttfamily,
  commentstyle=\color{codegreen}\ttfamily,
	stringstyle=\color{codepurple}\ttfamily\sffamily,
	captionpos=b,
	numbers=left,
	numberstyle=\footnotesize\color{codegray},
	stepnumber=1,
  numbersep=5pt,
  language=C
}

\lstdefinestyle{lstStylecpp}{%
  style=lstStyleBase,
  frame=l,
  rulecolor=\color{blue},
  language=C++
}

\lstdefinestyle{lstStyleverilog}{%
  style=lstStyleBase,
  frame=l,
  rulecolor=\color{brown},
  language=verilog
}

\lstdefinestyle{lstStylepython}{%
  style=lstStyleBase,
  frame=l,
  rulecolor=\color{pink},
  language=python
}

\lstnewenvironment{code}{\lstset{style=lstStyleBase}}{}
\lstnewenvironment{latex}{\lstset{style=lstStyleLaTeX}}{}
\lstnewenvironment{shell}{\lstset{style=lstStyleShell}}{}
\lstnewenvironment{cdisplay}{\lstset{style=lstStylecdisplay}}{}
\lstnewenvironment{cpp}{\lstset{style=lstStylecpp}}{}
\lstnewenvironment{verilog}{\lstset{style=lstStyleverilog}}{}
\lstnewenvironment{python}{\lstset{style=lstStylepython}}{}
\lstnewenvironment{cpseudo}{\lstset{style=lstStylecpseudo}}{}
%    \end{macrocode}
%
% \subsection{快速插入图片或图表}
%    \begin{macrocode}
\newcommand{\figoptadd}[2]{
	\begin{figure}[H]
		\centering
		\includegraphics[#1]{#2}
	\end{figure}
}

%%%%%%%%%%%%%%%%%%%%
\newcommand{\figoptaddcap}[3]{
	\begin{figure}[H]
		\centering
		\includegraphics[#1]{#2}
		\caption{#3}
		\label{fig:#2}
	\end{figure}
}
%%%%%%%%%%%%%%%%%%%
\newcommand{\tabncc}[3]{
	\begin{table}[H]
		\centering
		\begin{tabular}{|*{#1}{c|}}
		\toprule
		#2\\
		\bottomrule
	\end{tabular}
	\caption{#3}
	\label{form:#3}
\end{table}}
%%%%%%%%%%%%%%%%%%%
\newcommand{\tabnc}[2]{
	\begin{table}[H]
		\centering
		\begin{tabular}{|*{#1}{c|}}
		\toprule
		#2\\
		\bottomrule
	\end{tabular}
\end{table}}
\newcommand{\tnl}{\tabularnewline\midrule}
%    \end{macrocode}
%
% \subsection{借用dtx文件代码}
%    \begin{macrocode}
\def\cmd#1{\cs{\expandafter\cmd@to@cs\string#1}}
\def\cmd@to@cs#1#2{\char\number`#2\relax}
\DeclareRobustCommand\cs[1]{\texttt{\char`\\#1}}
\newcommand*{\meta}[1]{{%
  \ensuremath{\langle}\rmfamily\itshape#1\/\ensuremath{\rangle}}}
\providecommand\marg[1]{%
  {\ttfamily\char`\{}\meta{#1}{\ttfamily\char`\}}}
\providecommand\oarg[1]{%
  {\ttfamily[}\meta{#1}{\ttfamily]}}
\providecommand\parg[1]{%
  {\ttfamily(}\meta{#1}{\ttfamily)}}
\providecommand\pkg[1]{{\sffamily#1}}
%    \end{macrocode}
% 
% \subsection{水印}
%    \begin{macrocode}
\RequirePackage{watermark}
\ifnju@draft
\AtEndOfClass{
	\watermark{% 
		\parbox[b][\paperheight]{\paperwidth}{% 
		\vfill 
		\centering% 
		\begin{tikzpicture}[remember picture,overlay] 
			\node [rotate=45,scale=10] at ($(current page.center) +(-1cm,1cm)$) 
			{\textcolor[gray]{0.8}{DRAFT}}; 
			\node [rotate=45,scale=3] at ($(current page.center) +(1cm,-1cm)$) 
			{\textcolor[gray]{0.75}{Compile time: \the\year - \the\month - \the\day}}; 
		\end{tikzpicture}% 
		\vfill 
		}
  }
}
\fi
%    \end{macrocode}
% 补丁
%    \begin{macrocode}
\renewcommand{\thesection}{\arabic{section}}
\newcommand{\nchapter}[1]{
  {\let\clearpage\relax\par\vspace{1cm} \chapter*{\LARGE#1}}
}
\newcommand*{\rom}[1]{\expandafter\@slowromancap\romannumeral #1@}
%    \end{macrocode}
%
% \subsection{自定义代码}
%    \begin{macrocode}

\newcommand{\blankpage}{
	\clearpage
	\begin{titlepage}
		\null\vfil
		\begin{center}
			\textit{This page intentionally left blank.}
		\end{center}
	\end{titlepage}
}
\newcommand{\rmnum}[1]{\romannumeral #1}
\newcommand{\Rmnum}[1]{\expandafter\@slowromancap\romannumeral #1@}
%    \end{macrocode}
% \subsection{结束部分}
% \label{sec:finish}
%    \begin{macrocode}
\AtEndOfClass{\sloppy}
%    \end{macrocode}
%</cls> 
%
%
%
% \iffalse
%    \begin{macrocode}
%<*dtx-style>
\ProvidesPackage{dtx-style}
\RequirePackage{hypdoc}
\RequirePackage{ifthen}
\RequirePackage[UTF8,scheme=chinese]{ctex}
\RequirePackage{newpxtext}
\RequirePackage{newpxmath}
\RequirePackage[
  top=2.5cm, bottom=2.5cm,
  left=4cm, right=2cm,marginparwidth=2.6cm,marginparsep=3mm,
  headsep=3mm]{geometry}
\RequirePackage{array,longtable,booktabs}
\RequirePackage{listings}
\RequirePackage{fancyhdr}
\RequirePackage{xcolor}
\definecolor{codegreen}{rgb}{0,0.6,0}
\definecolor{codegray}{rgb}{0.5,0.5,0.5}
\definecolor{codepurple}{rgb}{0.58,0,0.82}
\definecolor{backcolour}{rgb}{0.95,0.95,0.92}
\newcommand{\red}[1]{\textcolor{red}{#1}}
\newcommand{\redoverlay}[2]{\textcolor<#2>{red}{#1}}
\newcommand{\green}[1]{\textcolor{green}{#1}}
\newcommand{\greenoverlay}[2]{\textcolor<#2>{green}{#1}}
\newcommand{\blue}[1]{\textcolor{blue}{#1}}
\newcommand{\blueoverlay}[2]{\textcolor<#2>{blue}{#1}}
\newcommand{\purple}[1]{\textcolor{purple}{#1}}
\newcommand{\cyan}[1]{\textcolor{cyan}{#1}}
\newcommand{\teal}[1]{\textcolor{teal}{#1}}
\RequirePackage{enumitem}
\RequirePackage{etoolbox}
\RequirePackage{metalogo}
\RequirePackage{mathtools}
\DeclarePairedDelimiter{\ceil}{\lceil}{\rceil}
\DeclarePairedDelimiter{\floor}{\lfloor}{\rfloor}
\DeclareMathOperator{\Hamilton}{\hat{H}} 
\ifthenelse{\equal{\@nameuse{g__ctex_fontset_tl}}{mac}}{%
  \xeCJKsetwidth{‘’“”}{1em}
}{}

\colorlet{nju@macro}{blue!60!black}
\colorlet{nju@env}{blue!70!black}
\colorlet{nju@option}{purple}
\patchcmd{\PrintMacroName}{\MacroFont}{\MacroFont\bfseries\color{nju@macro}}{}{}
\patchcmd{\PrintDescribeMacro}{\MacroFont}{\MacroFont\bfseries\color{nju@macro}}{}{}
\patchcmd{\PrintDescribeEnv}{\MacroFont}{\MacroFont\bfseries\color{nju@env}}{}{}
\patchcmd{\PrintEnvName}{\MacroFont}{\MacroFont\bfseries\color{nju@env}}{}{}

\def\DescribeOption{%
  \leavevmode\@bsphack\begingroup\MakePrivateLetters%
  \Describe@Option}
\def\Describe@Option#1{\endgroup
  \marginpar{\raggedleft\PrintDescribeOption{#1}}%
  \nju@special@index{option}{#1}\@esphack\ignorespaces}
\def\PrintDescribeOption#1{\strut \MacroFont\bfseries\sffamily\color{nju@option} #1\ }
\def\nju@special@index#1#2{\@bsphack
  \begingroup
    \HD@target
    \let\HDorg@encapchar\encapchar
    \edef\encapchar usage{%
      \HDorg@encapchar hdclindex{\the\c@HD@hypercount}{usage}%
    }%
    \index{#2\actualchar{\string\ttfamily\space#2}
           (#1)\encapchar usage}%
    \index{#1:\levelchar#2\actualchar
           {\string\ttfamily\space#2}\encapchar usage}%
  \endgroup
  \@esphack}

\lstdefinestyle{lstStyleBase}{%
   basicstyle=\small\ttfamily,
   aboveskip=\medskipamount,
   belowskip=\medskipamount,
   lineskip=0pt,
   boxpos=c,
   showlines=false,
   extendedchars=true,
   upquote=true,
   tabsize=2,
   showtabs=false,
   showspaces=false,
   showstringspaces=false,
   numbers=none,
   linewidth=\linewidth,
   xleftmargin=4pt,
   xrightmargin=0pt,
   resetmargins=false,
   breaklines=true,
   breakatwhitespace=false,
   breakindent=0pt,
   breakautoindent=true,
   columns=flexible,
   keepspaces=true,
   gobble=2,
   framesep=3pt,
   rulesep=1pt,
   framerule=1pt,
   backgroundcolor=\color{gray!5},
   stringstyle=\color{green!40!black!100},
   keywordstyle=\bfseries\color{blue!50!black},
   commentstyle=\slshape\color{black!60}}

\lstdefinestyle{lstStyleShell}{%
   style=lstStyleBase,
   frame=l,
   rulecolor=\color{purple},
   language=bash}

\lstdefinestyle{lstStyleLaTeX}{%
   style=lstStyleBase,
   frame=l,
   rulecolor=\color{violet},
   language=[LaTeX]TeX}
\lstdefinestyle{lstStylecpp}{%
   style=lstStyleBase,
   frame=l,
   rulecolor=\color{blue},
   language=C++
 }

\lstnewenvironment{latex}{\lstset{style=lstStyleLaTeX}}{}
\lstnewenvironment{shell}{\lstset{style=lstStyleShell}}{}
\lstnewenvironment{cpp}{\lstset{style=lstStylecpp}}{}

\setlist{nosep}

\DeclareDocumentCommand{\option}{m}{\textsf{#1}}
\DeclareDocumentCommand{\env}{m}{\texttt{#1}}
\DeclareDocumentCommand{\pkg}{s m}{%
  \texttt{#2}\IfBooleanF#1{\nju@special@index{package}{#2}}}
\DeclareDocumentCommand{\file}{s m}{%
  \texttt{#2}\IfBooleanF#1{\nju@special@index{file}{#2}}}
\newcommand{\myentry}[1]{%
  \marginpar{\raggedleft\color{purple}\bfseries\strut #1}}
\newcommand{\note}[2][Note]{{%
  \color{magenta}{\bfseries #1}\emph{#2}}}

\def\njurepo{\textsc{NJU}\-\textsc{repo}}
\def\njuthesis{\textsc{NJU}\-\textsc{Thesis}}
%</dtx-style>
%    \end{macrocode}
% \fi
% \Finale
