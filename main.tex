\documentclass[language=english,open=any]{njurepo}
\def\cmd#1{\cs{\expandafter\cmd@to@cs\string#1}}
\def\cmd@to@cs#1#2{\char\number`#2\relax}
\DeclareRobustCommand\cs[1]{\texttt{\char`\\#1}}

\newcommand*{\meta}[1]{{%
  \ensuremath{\langle}\rmfamily\itshape#1\/\ensuremath{\rangle}}}
\providecommand\marg[1]{%
  {\ttfamily\char`\{}\meta{#1}{\ttfamily\char`\}}}
\providecommand\oarg[1]{%
  {\ttfamily[}\meta{#1}{\ttfamily]}}
\providecommand\parg[1]{%
  {\ttfamily(}\meta{#1}{\ttfamily)}}
\providecommand\pkg[1]{{\sffamily#1}}
% 表格中支持跨行
\RequirePackage{multirow}

% 固定宽度的表格
\RequirePackage{tabularx}

% 表格中的反斜线
\RequirePackage{diagbox}

% 确定浮动对象的位置,可以使用 H,强制将浮动对象放到这里(可能效果很差)
\RequirePackage{float}

\begin{document}
\frontmatter
\njusetup{
  %******************************
  % 注意:
  %   1. 配置里面不要出现空行
  %   2. 不需要的配置信息可以删除
  %******************************
  %
  %=========
  % 中文信息
  %=========
  ctitle={清华大学学位论文 \LaTeX\ 模板\\使用示例文档},
  csubtitle={通用报告模板},
  cdegree={工学硕士},
  cdepartment={计算机科学与技术系},
  cmajor={计算机科学与技术},
  cauthor={薛瑞尼},
  stdid={12345678},
  csupervisor={郑纬民教授},
  cassosupervisor={陈文光教授}, % 副指导老师
  ccosupervisor={某某某教授}, % 联合指导老师
  % 日期自动使用当前时间,若需指定按如下方式修改:
  % cdate={超新星纪元},
  %
  %=========
  % 英文信息
  %=========
  etitle={An Introduction to \LaTeX{} Thesis Template of Tsinghua University v\version},
  esubtitle={subtile here},
  %esubsubtitle={Sub},
  edegree={Doctor of Engineering},
  emajor={Computer Science and Technology},
  eauthor={Xue Ruini},
  esupervisor={Professor Zheng Weimin},
  eassosupervisor={Chen Wenguang},
  % 日期自动生成,若需指定按如下方式修改:
  % edate={December, 2005}
  %
  % 关键词用“英文逗号”分割
  ckeywords={\TeX, \LaTeX, CJK, 模板, 论文},
  ekeywords={\TeX, \LaTeX, CJK, template, thesis}
}


\input{examples/abstract}
\makecover
\tableofcontents
\input{examples/denotation}
\mainmatter
\include{examples/chap01}
\include{examples/chap02}
%\include{examples/chap02}
\backmatter
\listoffigures
\listoftables
\listofequations
\bibliographystyle{numeric}
%\bibliographystyle{author-year}
\bibliography{ref/refs}
\include{examples/ack}
%% 附录
\begin{appendix}
  \input{examples/appendix01}
  \end{appendix}
\end{document}